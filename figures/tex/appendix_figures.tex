% appendix_figures.tex
% Core v2.5 — Appendix G: ASCII Figures and Structural Maps
%
% All diagrams are represented as ASCII art inside verbatim blocks.
% No additional packages (TikZ, listings) are required.

\section{Appendix G: ASCII Figures and Structural Maps}
\label{sec:appendix-figures}

This appendix collects ASCII diagrams that visualise the core
structures of the Continuum Framework.  
Each diagram is accompanied by a short caption linking it to the formal
definitions in the main text.

% -------------------------------------------------
% Figure G.1 — Universal Architecture of a Continuum
% -------------------------------------------------

\subsection*{Figure G.1: Universal Architecture of a Continuum}
\label{subsec:fig-architecture}

\begin{verbatim}
                      +-------------------+
                      |    boundary       |
                      |   dOmega(K)       |
                      +---------^---------+
                                |
                                | critical states
                                |
      +-----------+-------------+-------------+-----------+
      |           |                           |           |
      v           v                           v           v
   +-----+     +-----+       +-----------+  +-----+     +-----+
   |  A  |     |  P  |       |  Theta   |  |  J  |     |  C  |
   +-----+     +-----+       +-----------+  +-----+     +-----+
                      \         |   |        /
                       \        |   |       /
                        \       |   |      /
                         v      v   v     v

                       +-------------------+
                       |    Omega(K)       |
                       | admissible states |
                       +-------------------+

                               |
                               v
                           +-------+
                           |   k   |
                           | (K,t) |
                           +-------+
\end{verbatim}

\noindent
\textbf{Caption.}  
The admissible set \(\Omega(K)\) is determined by axes \(A\), potentials
\(P\), thresholds \(\Theta\), flows \(J\) and cycles \(C\).  
The boundary \(\partial\Omega(K)\) (here labelled \texttt{dOmega(K)})
marks the onset of critical states where thresholds are saturated.
The scalar \(k(K,t)\) summarises the effective continuumness at time
\(t\).

% -------------------------------------------------
% Figure G.2 — K-ladder (Hierarchy K0–K12)
% -------------------------------------------------

\subsection*{Figure G.2: K-ladder (Hierarchy K0–K12)}
\label{subsec:fig-k-ladder}

\begin{verbatim}
             meta-integrative
                  K12  (artificial / global)
                  |
                  v
             K11  (language / semantics)
                  |
                  v
             K10  (metamodels / categories)
                  |
                  v
             K9   (metatheory / paradigms)
                  |
                  v
             K8   (civilisations / technosphere)
                  |
                  v
             K7   (social systems / institutions)
                  |
                  v
             K6   (cognition / internal models)
                  |
                  v
             K5   (neural networks / spikes)
                  |
                  v
             K4   (protocells / membranes)
                  |
                  v
             K3   (autocatalytic chemistry)
                  |
                  v
             K2   (percolation / critical fields)
                  |
                  v
             K1   (finite lattices / local rules)
                  |
                  v
             K0   (metalogical layer)

Each arrow Kx -> Kx+1 is a dimensional transition generated by Psi_{x->x+1},
subject to the admissibility conditions defined in the core model.
\end{verbatim}

\noindent
\textbf{Caption.}  
The K-ladder illustrates the ordered sequence of continua from
\(K_0\) (metalogue) up to \(K_{12}\) (meta-integrative artificial and
global systems).  
Each step corresponds to a dimensional transition
\(\Psi_{x\rightarrow x+1}\) where a new axis is introduced and the
admissible region is expanded.

% -------------------------------------------------
% Figure G.3 — Omega(K) and its Boundary dOmega(K)
% -------------------------------------------------

\subsection*{Figure G.3: Admissible Region \texorpdfstring{$\Omega(K)$}{Omega(K)}
and Boundary \texorpdfstring{$\partial\Omega(K)$}{dOmega(K)}}
\label{subsec:fig-omega-boundary}

\begin{verbatim}
                    state space S(K)
     ---------------------------------------------------
     |                                                 |
     |      +-------------------------------+          |
     |      |           Omega(K)            |          |
     |      |   admissible configurations   |          |
     |      |                               |          |
     |      |    o   o          o           |          |
     |      |                               |          |
     |      +-------------------------------+          |
     |             ^                       ^           |
     |             |                       |           |
     |       s in dOmega(K)              s in Omega(K)|
     |                                                 |
     |   x   x   x   x   x   x   x   x   x   x   x     |
     |   forbidden states (T(s) > 0, thresholds        |
     |   exceeded, or domains of A/P violated)         |
     ---------------------------------------------------
\end{verbatim}

\noindent
\textbf{Caption.}  
The interior of \(\Omega(K)\) contains states with all potentials and
axes within admissible domains and structural tension below threshold.  
Points on the boundary \(\partial\Omega(K)\) are critical; small
perturbations can move them out of the admissible region.  
States marked with \texttt{x} lie outside \(\Omega(K)\) and correspond
to structurally forbidden configurations.

% -------------------------------------------------
% Figure G.4 — Generic Phase Transition
% -------------------------------------------------

\subsection*{Figure G.4: Generic Phase Transition of Continuumness}
\label{subsec:fig-phase-transition}

\begin{verbatim}
         k
         ^
         |                     *
         |                    *
         |                   *
         |                  *
         |                 *
         |                *
         |               *
         |              *
         |             *
         |            *
         |           *
         |__________*__________________________>  control parameter
                    |
                    |
                 p_c (critical threshold)

Below p_c: k = 0   (no coherent continuum, only fragments)
Above p_c: k > 0   (continuum exists with long-range coherence)

Examples:
- bond probability in percolation models
- osmotic or mechanical load on protocells
- excitation level in neural tissue
- social load or institutional stress in K7
- civilisational tension in K8
\end{verbatim}

\noindent
\textbf{Caption.}  
A generic phase transition where the order parameter \(k\) changes from
zero to non-zero at a critical threshold \(p_c\).  
Different levels instantiate different control parameters, but the
structural form of the transition is the same.

% -------------------------------------------------
% Figure G.5 — K3 -> K4: Protocellular Closure
% -------------------------------------------------

\subsection*{Figure G.5: Transition from Autocatalytic Chemistry (K3)
to Protocells (K4)}
\label{subsec:fig-k3-k4}

\begin{verbatim}
  (A) Open RAF-network in bulk solution (K3)

           reactants, catalysts, products
     -------------------------------------------------
     |   o--->o--->o        o--->o                    |
     |    \    ^   /        ^                         |
     |     \  / \ /        /                          |
     |      o   o   (RAF network, no boundary)        |
     -------------------------------------------------

  (B) RAF-network enclosed by a semi-permeable membrane (K4)

             outside                |      inside
                                    |
     -------------------------------+--------------------
     |   solutes, waste            |   o--->o--->o      |
     |   etc.                      |    \    ^   /      |
     |                             |     \  / \ /       |
     |                             |      o   o         |
     |                             |  (RAF + gradients) |
     -------------------------------  semi-permeable  ---

Key points:
- new axis: inside / outside
- appearance of stable gradients
- membrane thresholds (osmotic, mechanical) define dOmega(K4)
\end{verbatim}

\noindent
\textbf{Caption.}  
A RAF-network in an unbounded medium (K3) becomes a protocellular
continuum (K4) once a semi-permeable membrane encloses it and stable
gradients can be maintained.  
This introduces a new axis (inside/outside) and new thresholds for
membrane integrity.

% -------------------------------------------------
% Figure G.6 — Spike Cycle in K5
% -------------------------------------------------

\subsection*{Figure G.6: Neuronal Spike Cycle in K5}
\label{subsec:fig-spike-cycle}

\begin{verbatim}
   V(t)
    ^
    |                   /\
    |                  /  \
    |                 /    \
    |                /      \
    |   rest  _____ /        \_____  rest
    |             /            \
    |            /              \
    +----------------------------------------->  t

          ^                ^
          |                |
     Theta_spike      refractory return

Phases:
- rest: stable membrane potential
- depolarisation: V(t) crosses Theta_spike
- spike peak and repolarisation
- hyperpolarisation and return to rest

One full loop = spike cycle C_spike with period tau_spike.
\end{verbatim}

\noindent
\textbf{Caption.}  
The spike cycle is a minimal time-carrying loop in \(K_5\).  
Crossing the threshold \(\Theta_{\mathrm{spike}}\) corresponds to a
local phase transition; the closed loop defines a cycle
\(C_{\mathrm{spike}}\) with period \(\tau_{\mathrm{spike}}\).  
Sustained neural activity requires repeated traversal of this cycle.

% -------------------------------------------------
% Figure G.7 — Institutional Cycle in K7
% -------------------------------------------------

\subsection*{Figure G.7: Institutional Cycle in the Social Continuum K7}
\label{subsec:fig-inst-cycle}

\begin{verbatim}
                 +-----------------------+
                 |   norm formation      |
                 +-----------+-----------+
                             |
                             v
                 +-----------------------+
                 |    codification       |
                 |  (laws, rules, etc.) |
                 +-----------+-----------+
                             |
                             v
                 +-----------------------+
                 |    implementation     |
                 | (organisations, roles)|
                 +-----------+-----------+
                             |
                             v
                 +-----------------------+
                 |    enforcement        |
                 | (sanctions, rewards)  |
                 +-----------+-----------+
                             |
                             v
                 +-----------------------+
                 |      feedback         |
                 | (legitimacy, trust)  |
                 +-----------+-----------+
                             |
                             v
                 +-----------------------+
                 |      revision         |
                 +-----------+-----------+
                             |
                             v
                 +-----------------------+
                 |   norm formation      |
                 +-----------------------+

If any arrow is broken, the cycle C_inst fails and the institution
leaves Omega(K7), contributing to k7 -> 0.
\end{verbatim}

\noindent
\textbf{Caption.}  
Institutions are represented as cycles \(C_{\mathrm{inst}}\) in the
social continuum \(K_7\).  
Loss of any stage in the loop (formation, codification, implementation,
enforcement, feedback, revision) leads to institutional breakdown and a
reduction of \(k_7\).

% -------------------------------------------------
% Figure G.8 — Civilisational Tension in K8
% -------------------------------------------------

\subsection*{Figure G.8: Civilisational Tension and Stability in K8}
\label{subsec:fig-civ-tension}

\begin{verbatim}
    T8
    ^                         /
    |                        /
    |                       /
    |             _________/      collapse / reorganisation
    |            /
    |           /
    |          /
    |         /
    |________/______________________________>  time or load

              ^    ^
              |    |
        low tension   high tension
        stable cycles C8
        -> k8 ~ 1

Theta_stab marks the boundary:
- below Theta_stab: civilisational continuum K8 can re-stabilise
- above Theta_stab: cycles C8 fail, k8 decreases, partial or total
  collapse occurs

Hysteresis:
- reducing load after collapse does not automatically restore
  the original k8.
\end{verbatim}

\noindent
\textbf{Caption.}  
Civilisational tension \(T_8\) increases under external and internal
stress.  
Once it exceeds the stability threshold \(\Theta_{\mathrm{stab}}\),
key cycles \(C_8\) fail and the civilisational continuum leaves its
admissible region \(\Omega(K_8)\).  
The resulting dynamics exhibit hysteresis: restoring previous
conditions does not guarantee recovery of the original \(k_8\).

% -------------------------------------------------
% Summary
% -------------------------------------------------

\section*{Summary of ASCII Figures}
\label{sec:appendix-figures-summary}

The ASCII diagrams in this appendix visualise:
\begin{itemize}
  \item the universal architecture of a continuum,
  \item the K-ladder from \(K_0\) to \(K_{12}\),
  \item the admissible region \(\Omega(K)\) and its boundary,
  \item generic threshold transitions of continuumness,
  \item the protocellular transition \(K_3 \rightarrow K_4\),
  \item neuronal spike cycles in \(K_5\),
  \item institutional cycles in \(K_7\),
  \item and civilisational tension in \(K_8\).
\end{itemize}

They are designed to be fully compatible with automated LaTeX builds
and to provide robust visual intuition without additional graphics
packages.
