\section{Introduction}

Scientific models across physics, chemistry, biology and social systems rely on
domain-specific assumptions and formalisms. Despite their differences, these
domains exhibit structurally identical phenomena: the formation of stable
regimes, the emergence of new degrees of freedom, threshold-driven transitions,
and collapse under constraint violation. Current theories describe these
processes locally but lack a unified formal basis.

The Ontology of Continua (OC) introduces such a basis. A continuum is defined as
a structured system whose existence depends on the persistence of an admissible
state domain under dynamic constraints. A continuum is characterised by a compact
set of components:
\begin{itemize}
    \item axes of distinction $A$,
    \item potentials $P$,
    \item thresholds $\Theta$,
    \item fluxes $J$,
    \item cycles $C$,
    \item admissible domain $\Omega(K)$,
    \item boundary $\partial\Omega(K)$,
    \item structural tension $T$,
    \item continuity measure $k(t)$.
\end{itemize}

These components are combined by a universal operator
\[
\frac{dX_K}{dt} = (F,G,H,Q,R,S,U)(X_K,M),
\]
which governs the evolution of any continuum $K$ inside a metaspace $M$. The
operator formalises the conditions under which a continuum persists, expands its
dimensionality, or collapses.

OC provides a hierarchical structure $K_0$--$K_{12}$ describing how continua
emerge through successive increases in structural complexity. The hierarchy
spans proto-continua, physical fields, chemical reaction networks, protocells,
bioelectric systems, cognitive architectures, social institutions,
civilisational systems and formal-recursive metatheories. Each transition
$K_i \rightarrow K_{i+1}$ is defined by the appearance of a new axis of
distinction triggered when structural tension exceeds a critical threshold.

The objective of OC is to supply a unified axiomatic framework capable of
expressing emergence, stability, criticality and collapse across scientific
domains. This preprint presents the core axioms, the universal operator, the
K-level hierarchy, the formal continuity conditions, and the associated
falsifiability criteria.
