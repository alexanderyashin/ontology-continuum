% appendix_dynamics.tex
% Core v2.5 — Appendix F: Dynamics, Operators, and Evolution Equations

\section{Appendix F: Dynamics and Evolution Equations}
\label{sec:appendix-dynamics}

This appendix provides the complete formal structure of dynamical evolution for
all continua \(K_0,\dots,K_{12}\). The evolution of a continuum is defined by a
universal operator
\[
  E : K(t) \longrightarrow K(t+\Delta t),
\]
which decomposes into component-wise laws acting on the tuple
\(
  (\Omega, A, P, \Theta, J, C, k, T).
\)

\subsection{Universal Evolution Scheme}
\label{subsec:appendix-dynamics-universal}

The fundamental dynamical system is:
\begin{align*}
  \frac{dA}{dt} &= F(A,P,\Theta,J,C,t), \\
  \frac{dP}{dt} &= G(A,P,\Theta,J,C,t), \\
  \frac{d\Theta}{dt} &= H(A,P,\Theta,J,C,t), \\
  \frac{dT}{dt} &= Q(A,P,\Theta,J,C,T,t), \\
  \frac{dJ}{dt} &= S(A,P,\Theta,J,C,t), \\
  \frac{dC}{dt} &= R(A,P,\Theta,J,C,T,t), \\
  \frac{dk}{dt} &= U(A,P,\Theta,J,C,T,k,t).
\end{align*}

The operators depend on the current level \(K_x\), its metaspace \(M_{x+1}\),
and the admissibility region \(\Omega_x\).  
All components are coupled through tension and thresholds.

\subsection{Constraints for Dynamics}
\label{subsec:appendix-dynamics-constraints}

Each operator satisfies the following structural constraints:

\paragraph{(Dyn1) Admissibility preservation.}
\[
  T(s(t))\le 0 \;\Rightarrow\; s(t+\Delta t)\in\Omega(K).
\]

\paragraph{(Dyn2) Boundary saturation constraint.}
States may approach but not cross the admissibility boundary while \(k>0\):
\[
  T(t)=0 \Rightarrow T(t+\Delta t)\le 0.
\]

\paragraph{(Dyn3) Dimension monotonicity.}
A new axis may appear only if
\[
  T \ge \Theta_{\mathrm{dim}},
\]
and existing axes cannot disappear unless \(k=0\).

\paragraph{(Dyn4) Cycle continuity.}
Existing cycles deform continuously or merge into new ones;  
abrupt loss of cycles implies collapse.

\paragraph{(Dyn5) Metaspace compatibility.}
\[
  \Omega(K_x(t))\subseteq\Omega(M_x)
  \quad\forall t.
\]

\subsection{Operator \(F\): Axis Dynamics}
\label{subsec:appendix-dynamics-F}

Axis evolution is defined by:
\[
  \frac{dA_j}{dt} = F_j(A,P,\Theta,J,C,t).
\]

Examples:
\begin{itemize}
  \item \(K_2\): topological class of cluster connectivity.
  \item \(K_4\): membrane curvature/permeability axes.
  \item \(K_6\): activation of cognitive axes (binding, prediction).
  \item \(K_7\): emergence or decay of social norm axes.
  \item \(K_8\): technological/institutional axes.
\end{itemize}

\subsection{Operator \(G\): Potential Dynamics}
\label{subsec:appendix-dynamics-G}

Potentials evolve via:
\[
  \frac{dP_i}{dt} = G_i(A,P,\Theta,J,C,t).
\]

Examples:
\begin{itemize}
  \item \(K_3\): chemical potentials driven by reaction fluxes.
  \item \(K_5\): voltage and ionic gradients.
  \item \(K_6\): prediction-error dynamics.
  \item \(K_7\): conflict/trust potentials.
  \item \(K_8\): systemic energy/information potentials.
\end{itemize}

\subsection{Operator \(H\): Threshold Dynamics}
\label{subsec:appendix-dynamics-H}

Slow adaptation of thresholds:
\[
  \frac{d\Theta_\ell}{dt} = H_\ell(A,P,\Theta,J,C,t).
\]

Examples:
\begin{itemize}
  \item \(K_4\): mechanical/energetic membrane constraints.
  \item \(K_6\): predictive tolerance thresholds.
  \item \(K_7\): institutional and cultural stability limits.
  \item \(K_8\): infrastructure load thresholds.
\end{itemize}

\subsection{Operator \(Q\): Tension Dynamics}
\label{subsec:appendix-dynamics-Q}

Tension evolves through:
\[
  \frac{dT}{dt} = Q(A,P,\Theta,J,C,T,t),
\]
with:
\[
  T(s)\le 0 \;\Rightarrow\; s\in\Omega(K),\qquad
  T(s)=0\;\Rightarrow\; s\in\partial\Omega(K).
\]

Examples:
\begin{itemize}
  \item \(K_3\): catalytic deficit vs. activation barrier.
  \item \(K_4\): osmotic stress vs. membrane tension.
  \item \(K_6\): model-inconsistency tension.
  \item \(K_7\): institutional/coordination stress.
\end{itemize}

\subsection{Operator \(S\): Flow Dynamics}
\label{subsec:appendix-dynamics-S}

Flows evolve by:
\[
  \frac{dJ_m}{dt} = S_m(A,P,\Theta,J,C,t).
\]

Examples:
\begin{itemize}
  \item \(K_3\): reaction flux modulation.
  \item \(K_4\): ion/osmotic fluxes.
  \item \(K_5\): synaptic/neuronal currents.
  \item \(K_7\): communication or influence flows.
\end{itemize}

\subsection{Operator \(R\): Cycle Dynamics}
\label{subsec:appendix-dynamics-R}

Cycles obey:
\[
  \frac{dC_n}{dt} = R_n(A,P,\Theta,J,C,T,t).
\]

Examples:
\begin{itemize}
  \item \(K_3\): metabolic closure maintenance.
  \item \(K_5\): rhythmic/pattern-forming cycles.
  \item \(K_6\): attractor restructuring.
  \item \(K_7\): institutional reinforcement loops.
\end{itemize}

\subsection{Operator \(U\): Continuumness Dynamics}
\label{subsec:appendix-dynamics-U}

Continuumness evolves as:
\[
  \frac{dk}{dt} = U(A,P,\Theta,J,C,T,k,t).
\]

With:
\[
  U>0 \;\text{when resilience increases},\qquad
  U<0 \;\text{approaching thresholds}.
\]

Examples:
\begin{itemize}
  \item \(K_2\): likelihood of infinite-cluster formation.
  \item \(K_4\): stability of membrane + metabolic subsystems.
  \item \(K_6\): coherence of cognitive models.
  \item \(K_7\): social cohesion dynamics.
\end{itemize}

\subsection{Parameter Dynamics}
\label{subsec:appendix-dynamics-params}

General parameter evolution:
\[
  \frac{d\Pi}{dt} = W(A,P,\Theta,J,C,T,k,\Pi,t).
\]

This captures learning (K6), genetic drift (K4–K5), institutional drift (K7),
technological change (K8) and model drift (K11).

\subsection{Dynamical Stability}
\label{subsec:appendix-dynamics-stability}

Define the landscape:
\[
  L(s)=T(s)-\Theta(s).
\]

Stability requires:
\[
  \frac{dL}{dt}\le 0.
\]

Equivalently:
\[
  T(t)\le\Theta(t)\quad\forall t.
\]

\subsection{Cycle Formation and Destruction}
\label{subsec:appendix-dynamics-cycles}

Cycles form when:
\[
  R_n>0 \quad\text{and}\quad C_n(t)\neq\varnothing.
\]

Cycles collapse when:
\[
  C_n=\varnothing
  \;\Longleftrightarrow\;
  T>\Theta_{\mathrm{cycle}}.
\]

Loss of all cycles implies:
\[
  C=\varnothing \Rightarrow k=0.
\]

\subsection{Dimension Change under Dynamics}
\label{subsec:appendix-dynamics-dim}

Dimension increases when:
\[
  T\ge\Theta_{\mathrm{dim}}.
\]
Dimension cannot decrease unless \(k=0\).

\subsection{Time Continuity}
\label{subsec:appendix-dynamics-time}

The characteristic time scale satisfies:
\[
  \tau(K)=\min_n \tau_n.
\]

Critical slowing down:
\[
  \tau\to+\infty \;\Rightarrow\; k\to 0.
\]

\subsection{Metaspace Constraints}
\label{subsec:appendix-dynamics-meta}

Dynamics respect:
\[
  \Omega(K_x)\subseteq\Omega(M_x),\qquad
  A(K_x)\subseteq A(M_x).
\]

\subsection{Compact Summary}
\label{subsec:appendix-dynamics-summary}

This appendix formalises:

\begin{itemize}
  \item the universal dynamic operator \(E\),
  \item component laws \(F,G,H,Q,S,R,U\),
  \item parameter evolution \(W\),
  \item tension-driven constraints,
  \item metaspace compatibility,
  \item cycle dynamics and collapse rules,
  \item dimension growth and time continuity.
\end{itemize}

Together, these equations unify the dynamics of all continua
\(K_0\) through \(K_{12}\) under a single coherent formal system.
