% appendix_dynamics.tex
% Core v2.5 — Appendix F: Dynamics, Operators, and Evolution Equations

\section{Appendix F: Dynamics and Evolution Equations}
\label{sec:appendix-dynamics}

This appendix gives the full formalisation of dynamical equations,
operators, and structural constraints that govern the evolution of all
continua \(K_0,\dots,K_{12}\).
Every continuum evolves according to the universal operator
\[
  E : K(t) \longrightarrow K(t+\Delta t),
\]
with component-wise update laws. These laws are specified below.

\subsection{Universal Evolution Scheme}
\label{subsec:appendix-dynamics-universal}

For any continuum \(K\), let the components be
\(
  (\Omega, A, P, \Theta, J, C, k)
\).
Their time evolution is defined by:
\begin{align*}
  \frac{dA}{dt} &= F(A,P,\Theta,J,C,t), \\
  \frac{dP}{dt} &= G(A,P,\Theta,J,C,t), \\
  \frac{d\Theta}{dt} &= H(A,P,\Theta,J,C,t), \\
  \frac{dJ}{dt} &= Q(A,P,\Theta,J,C,t), \\
  \frac{dC}{dt} &= S(A,P,\Theta,J,C,t), \\
  \frac{dk}{dt} &= U(A,P,\Theta,J,C,k,t).
\end{align*}

The operators \(F,G,H,Q,S,U\) depend on the level \(K_x\), its metaspace
\(M_{x+1}\), and the local admissibility conditions \(\Omega_x\).

\subsection{Constraints for Dynamics}
\label{subsec:appendix-dynamics-constraints}

Each operator must satisfy:

\paragraph{(Dyn1) Admissibility Preservation.}
If \(A(t)\in \mathcal{A}_{A_j}\), then
\[
  A(t+\Delta t) \in \mathcal{A}_{A_j}.
\]

\paragraph{(Dyn2) Threshold Compliance.}
The dynamics may approach \(\partial\Omega\), but cannot move past it
while \(k>0\):
\[
  T(t) \le 0 \Rightarrow T(t+\Delta t) \le 0.
\]

\paragraph{(Dyn3) Monotonic Dimension Constraint.}
New axes may appear only if
\[
  T \ge \Theta_{\mathrm{dim}},
\]
and may never disappear unless \(k=0\).

\paragraph{(Dyn4) Cycle Preservation.}
Cycles must either persist or transform continuously into new cycles.

\paragraph{(Dyn5) Metaspace Compatibility.}
\[
  \Omega(K_x(t)) \subseteq \Omega(M_x)
  \quad\text{for all } t.
\]

\subsection{Operator \(F\): Axis Dynamics}
\label{subsec:appendix-dynamics-F}

Axis evolution governs changes in the coordinates that define distinctions:
\[
  \frac{dA_j}{dt} = F_j(A,P,\Theta,J,C,t).
\]

Examples:
\begin{itemize}
  \item \(K_2\): change in cluster-connectivity class.
  \item \(K_4\): time evolution of membrane permeability axis.
  \item \(K_6\): activation/deactivation of cognitive axes.
  \item \(K_7\): emergence or disappearance of social norms/roles.
  \item \(K_8\): emergence of new technological or institutional axes.
\end{itemize}

\subsection{Operator \(G\): Potential Dynamics}
\label{subsec:appendix-dynamics-G}

Potentials evolve by:
\[
  \frac{dP_i}{dt} = G_i(A,P,\Theta,J,C,t).
\]

Examples:
\begin{itemize}
  \item \(K_3\): change in chemical potentials due to fluxes.
  \item \(K_5\): change in membrane potentials due to ion currents.
  \item \(K_6\): prediction-error dynamics.
  \item \(K_7\): trust/conflict potentials in social networks.
\end{itemize}

\subsection{Operator \(H\): Threshold Dynamics}
\label{subsec:appendix-dynamics-H}

Thresholds evolve slowly:
\[
  \frac{d\Theta_\ell}{dt} = H_\ell(A,P,\Theta,J,C,t).
\]

Examples:
\begin{itemize}
  \item \(K_4\): mechanical limits of membranes adjusting to stress.
  \item \(K_6\): adjustment of prediction-error tolerance.
  \item \(K_7\): change in institutional stability thresholds.
  \item \(K_8\): systemic thresholds tied to infrastructure capacity.
\end{itemize}

\subsection{Operator \(Q\): Flow Dynamics}
\label{subsec:appendix-dynamics-Q}

Flows evolve via:
\[
  \frac{dJ_m}{dt} = Q_m(A,P,\Theta,J,C,t).
\]

Examples:
\begin{itemize}
  \item \(K_3\): reaction flux modifications.
  \item \(K_4\): osmotic/electrochemical fluxes.
  \item \(K_5\): neuronal currents and synaptic fluxes.
  \item \(K_7\): communication/influence flows.
  \item \(K_8\): energy/resource flows.
\end{itemize}

\subsection{Operator \(S\): Cycle Dynamics}
\label{subsec:appendix-dynamics-S}

Cycle evolution is:
\[
  \frac{dC_n}{dt} = S_n(A,P,\Theta,J,C,t).
\]

Examples:
\begin{itemize}
  \item \(K_3\): metabolic closure or cycle amplification.
  \item \(K_5\): formation/destruction of rhythmic firing patterns.
  \item \(K_6\): restructuring of model-attractor cycles.
  \item \(K_7\): reinforcement or collapse of institutional cycles.
  \item \(K_8\): long-wave stabilisation or destabilisation.
\end{itemize}

\subsection{Operator \(U\): Continuumness Dynamics}
\label{subsec:appendix-dynamics-U}

Continuumness evolves under:
\[
  \frac{dk}{dt} = U(A,P,\Theta,J,C,k,t),
\]
with:
\[
  U > 0 \quad\text{if stability improves},\qquad
  U < 0 \quad\text{if thresholds are approached}.
\]

Examples:
\begin{itemize}
  \item \(K_2\): increase of infinite-cluster probability.
  \item \(K_4\): stabilisation of membrane and metabolic loops.
  \item \(K_6\): increased coherence of cognitive models.
  \item \(K_7\): social cohesion gain.
  \item \(K_8\): infrastructural stabilisation.
\end{itemize}

\subsection{Parameter Dynamics}
\label{subsec:appendix-dynamics-params}

Parameters obey:
\[
  \frac{d\Pi}{dt} = W(A,P,\Theta,J,C,k,\Pi,t).
\]

This captures learning (K6), evolution (K4–K5), institutional drift (K7),
technological change (K8), and semantic drift (K11).

\subsection{Dynamical Stability}
\label{subsec:appendix-dynamics-stability}

A continuum is dynamically stable if:
\[
  \frac{dL}{dt} \le 0,
\]
where \(L(s) = T(s) - \Theta(s)\) is the landscape function.

Equivalently:
\[
  T(t) \le \Theta(t) \quad\forall t.
\]

\subsection{Cycle Formation and Destruction}
\label{subsec:appendix-dynamics-cycles}

Cycles form when:
\[
  S_n > 0,\quad
  C_n(t) \neq \varnothing.
\]

Cycles break when:
\[
  C_n = \varnothing
  \quad\Longleftrightarrow\quad
  T > \Theta_{\mathrm{cycle}}.
\]

Loss of all cycles implies death:
\[
  C=\varnothing \quad\Rightarrow\quad k=0.
\]

\subsection{Dimension Change under Dynamics}
\label{subsec:appendix-dynamics-dim}

Dimension increases when:
\[
  T \ge \Theta_{\mathrm{dim}}.
\]

Dimension cannot decrease unless \(k=0\).

\subsection{Continuity of Time}
\label{subsec:appendix-dynamics-time}

The time scale
\[
  \tau(K) = \min_n \tau_n
\]
evolves continuously except at collapse.

Near critical thresholds,
\[
  \tau \to +\infty
  \quad\Rightarrow\quad
  k \to 0.
\]

\subsection{Metaspace Constraints}
\label{subsec:appendix-dynamics-meta}

Dynamics must honour:
\[
  \Omega(K_x) \subseteq \Omega(M_x),\qquad
  A(K_x) \subseteq A(M_x),
\]
ensuring cross-level compatibility.

\subsection{Compact Summary}
\label{subsec:appendix-dynamics-summary}

This appendix formalises all dynamical components:

\begin{itemize}
  \item universal evolution operator \(E\),
  \item component dynamics \(F,G,H,Q,S,U\),
  \item parameter dynamics \(W\),
  \item constraints from threshold landscapes and metaspace,
  \item cycle evolution and dimension growth,
  \item universal stability and collapse conditions.
\end{itemize}

These equations unify the dynamics of all levels
\(K_0\) through \(K_{12}\)
under a single coherent structure.
