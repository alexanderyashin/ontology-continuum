% disciplines.tex
% Core v2.5 — Disciplines and Domains of Application

\section{Disciplines and Domains of Application}
\label{sec:disciplines}

The Continuum Framework is not a theory of a single field, but a
cross-level architecture intended to connect physics, chemistry,
biology, cognition, social systems, civilisations, and metatheory
under one structural template.  
In this section we map the levels
\[
  K_0,\dots,K_{12}
\]
to concrete scientific and philosophical disciplines and explain how
existing theories can be embedded into the continuum language.

The goal is not to replace established models, but to provide a
meta-structure in which they become comparable, composable, and
explicitly connected through the notions of:
\begin{itemize}
  \item admissible states \(\Omega(K_x)\),
  \item axes of difference \(A(K_x)\),
  \item potentials \(P(K_x)\),
  \item thresholds \(\Theta(K_x)\),
  \item flows \(J(K_x)\),
  \item cycles \(C(K_x)\),
  \item and continuumness \(k_x(t)\).
\end{itemize}

\subsection{Physics: Lattices, Fields and Percolation (\texorpdfstring{$K_0$--$K_2$}{K0--K2})}
\label{subsec:disciplines-physics}

The base levels of the Framework correspond to the traditional domain
of theoretical physics and mathematical foundations.

\paragraph{Foundations and mathematical physics (\(K_0\)).}
The metalogical layer \(K_0\) is not a physical system itself, but the
formal arena in which the conditions of possibility for physical
theories are formulated: logic, set theory, topology, measure theory
and the definition of state spaces and observables.  
This is the natural home of:
\begin{itemize}
  \item axiomatic formulations of probability and measure,
  \item the construction of configuration spaces and phase spaces,
  \item functional analytic foundations of quantum and classical fields.
\end{itemize}
In the continuum language, these structures define the ambient
spaces \(S(K_x)\) and the admissible domains for potentials and axes.

\paragraph{Lattice systems and local interactions (\(K_1\)).}
The level \(K_1\) captures finite systems with local interactions on a
discrete support, such as:
\begin{itemize}
  \item Ising and Potts models on finite lattices,
  \item cellular automata with local update rules,
  \item finite-network models in statistical mechanics.
\end{itemize}
Here, disciplines such as statistical mechanics and computational
physics appear in their most concrete form: the dynamics of
occupation variables, spins, or local fields on finite graphs.

\paragraph{Percolation and critical phenomena (\(K_2\)).}
The level \(K_2\) corresponds to the thermodynamic limit and to
continuum descriptions of critical behaviour, including:
\begin{itemize}
  \item percolation theory on infinite lattices,
  \item renormalisation-group descriptions near critical points,
  \item field-theoretic descriptions of topological phase transitions.
\end{itemize}
The function \(k_2(p)\), the probability that a site belongs to the
infinite cluster, becomes a prototype of an order parameter.  
The general principle is that many physical theories can be rewritten
as statements about how \(k_2\) changes when potentials and control
parameters cross critical thresholds.

\subsection{Chemistry and the Origin of Life (\texorpdfstring{$K_3$--$K_4$}{K3--K4})}
\label{subsec:disciplines-chemistry}

Levels \(K_3\) and \(K_4\) are associated with physical chemistry,
systems chemistry, and origin-of-life research.

\paragraph{Autocatalytic networks and systems chemistry (\(K_3\)).}
In \(K_3\) the central objects are reaction graphs and concentration
vectors. This is the domain of:
\begin{itemize}
  \item chemical reaction network theory,
  \item autocatalytic set theory and RAF networks,
  \item non-equilibrium thermodynamics in open reactors.
\end{itemize}
The continuum structure provides a systematic way to distinguish
between:
\begin{itemize}
  \item transient collections of reactions that die out, and
  \item self-sustaining, strongly connected autocatalytic networks
        that define a non-zero \(k_3\).
\end{itemize}
The notion of cycles \(C_3\) captures metabolic and catalytic loops
that remain active in admissible states.

\paragraph{Protocells and membrane-bound systems (\(K_4\)).}
Level \(K_4\) corresponds to protocells, vesicles and early cellular
systems, combining:
\begin{itemize}
  \item membrane biophysics,
  \item osmotic and mechanical stability,
  \item compartmentalised metabolism.
\end{itemize}
Disciplines here include soft-matter physics, biophysics of membranes,
and systems biology of minimal cells.  
From the continuum perspective, the appearance of a semi-permeable
boundary and stable gradients is a dimensional transition: a new axis
(inside/outside) turns previously unbounded chemistry into a
protocellular continuum with its own admissible region \(\Omega(K_4)\)
and non-zero \(k_4(t)\).

\subsection{Neuroscience and Cognition (\texorpdfstring{$K_5$--$K_6$}{K5--K6})}
\label{subsec:disciplines-neuro-cog}

Levels \(K_5\) and \(K_6\) are the domain of neuroscience,
computational neuroscience and cognitive science.

\paragraph{Neuronal dynamics (\(K_5\)).}
In \(K_5\) the elementary objects are neurons, ion channels and
synapses. Disciplines include:
\begin{itemize}
  \item electrophysiology (spike generation and propagation),
  \item network neuroscience,
  \item models of rhythmic activity and oscillations.
\end{itemize}
The continuum perspective emphasises spike cycles
\(C_{\mathrm{spike}}\), phase transitions at excitation thresholds,
and the emergence of network-level cycles that stabilise patterns of
activity. The degree of continuumness \(k_5\) reflects how much of the
network participates in coherent, sustainable dynamics.

\paragraph{Cognitive models and predictive processing (\(K_6\)).}
At level \(K_6\) the basic objects are internal models, beliefs and
prediction-error dynamics. This is the home of:
\begin{itemize}
  \item predictive processing and active inference,
  \item attractor network models of memory and decision-making,
  \item formal theories of representation and belief revision.
\end{itemize}
The axes \(A^{(6)}\) encode distinctions between models; the potentials
\(P^{(6)}\) quantify prediction error, value and salience; thresholds
\(\Theta_{\mathrm{cog}}\) determine when models must update or be
replaced.  
The Framework allows disparate cognitive models to be compared by
asking how they implement flows \(J_6\), cycles \(C_6\) and the
evolution of \(k_6(t)\) under perturbations.

\subsection{Social Systems and Institutions (\texorpdfstring{$K_7$}{K7})}
\label{subsec:disciplines-social}

Level \(K_7\) corresponds to sociology, political science, economics in
its institutional and organisational aspects, and parts of anthropology.

\paragraph{Communication, norms and roles.}
The state space of \(K_7\) includes agents, roles and communication
structures. This provides a unifying language for:
\begin{itemize}
  \item theories of social systems and communication,
  \item models of cooperation, norm formation and conflict,
  \item institutional analysis (law, organisations, governance).
\end{itemize}
The Framework treats social order not as a static structure but as a
network of cycles \(C_{\mathrm{inst}}\) in which norms are created,
codified, enforced, internalised and revised.  
Breakdown of an institution corresponds to leaving \(\Omega(K_7)\)
when thresholds for cooperation, legitimacy or enforcement are
violated. The function \(k_7\) summarises the effective connectedness
and stability of the social continuum.

\subsection{Civilisations and Technological Systems (\texorpdfstring{$K_8$}{K8})}
\label{subsec:disciplines-civilisation}

Level \(K_8\) belongs to macro-sociology, economic history, world
systems theory, technology studies and infrastructure research.

\paragraph{Technosphere and long-term dynamics.}
Here the state space includes technologies, infrastructures, economic
and legal systems, and symbolic structures such as writing and money.
Disciplines and topics include:
\begin{itemize}
  \item theories of civilisational rise and collapse,
  \item technology and innovation studies,
  \item large-scale infrastructure and network resilience,
  \item long-wave economic and political cycles.
\end{itemize}
The continuum lens emphasises how civilisations maintain a non-zero
\(k_8\) by stabilising crucial cycles \(C_8\) (for example food
production, energy supply, institutional reproduction).  
Thresholds \(\Theta_{\mathrm{stab}}\) mark the limits of load and
stress that these structures can absorb before systemic reorganisation
or collapse becomes unavoidable.

\subsection{Metatheory, Semantics and Artificial Systems (\texorpdfstring{$K_9$--$K_{12}$}{K9--K12})}
\label{subsec:disciplines-meta-ai}

The upper levels of the Framework connect to philosophy of science,
logic, linguistics and artificial intelligence.

\paragraph{Metatheoretical structures (\(K_9\)).}
Level \(K_9\) hosts:
\begin{itemize}
  \item philosophy of science,
  \item formal metatheory and model theory,
  \item studies of scientific paradigms and research programmes.
\end{itemize}
Theories, logics and models become elements of \(\Omega(K_9)\), and
their compatibility is analysed using axes of abstraction and
interpretation.  
Continuumness \(k_9\) reflects the coherence, explanatory power and
empirical support of a theoretical landscape.

\paragraph{Metamodels and categorical structure (\(K_{10}\)).}
Level \(K_{10}\) corresponds to category-theoretic approaches to
physics, computation and systems, as well as to general frameworks for
comparing models across disciplines.  
Objects are model categories; morphisms are functors that translate
between them.  
This is where work in category theory, universal algebra and abstract
semantics finds a natural representation.

\paragraph{Semantics and language (\(K_{11}\)).}
In \(K_{11}\) the central objects are meaning assignments, semantic
spaces and interpretative schemes. Disciplines include:
\begin{itemize}
  \item formal semantics and pragmatics,
  \item computational linguistics,
  \item distributional and geometric models of meaning.
\end{itemize}
The Framework describes semantic stability and drift in terms of
thresholds \(\Theta_{\mathrm{sem}}\), flows of interpretation \(J_{11}\)
and cycles of usage and re-interpretation \(C_{11}\).

\paragraph{Artificial systems and global integration (\(K_{12}\)).}
Level \(K_{12}\) links:
\begin{itemize}
  \item artificial intelligence and machine learning,
  \item multi-agent systems,
  \item information theory and algorithmic complexity.
\end{itemize}
Artificial systems can be treated as continua in their own right or as
components of larger human–machine continua.  
The meta-integrative role of \(K_{12}\) is to describe how natural and
artificial systems can be embedded into one global integrative space
with its own coherence measure \(k_{12}\).

\subsection{Cross-Disciplinary Patterns}
\label{subsec:disciplines-patterns}

Although the underlying phenomena differ dramatically between levels,
the Continuum Framework identifies a small set of structural motifs
that recur across disciplines:
\begin{itemize}
  \item \textbf{Thresholds and phase transitions:}
        critical values at which new axes appear or continua collapse
        (percolation thresholds, membrane-rupture limits, spike
        thresholds, cooperation thresholds, civilisational stress
        thresholds).
  \item \textbf{Cycles as carriers of time and stability:}
        from percolation correlation cycles through metabolic cycles,
        spike trains and institutional cycles to long civilisational
        waves.
  \item \textbf{Continuumness \(k(t)\) as a unifying order parameter:}
        measuring how much of the system participates in coherent,
        admissible dynamics at each level.
\end{itemize}

In this sense, the Framework provides a disciplined way of importing
and exporting ideas between fields: a result about the behaviour of
\(k_2\) near a critical threshold can have analogues in the behaviour
of \(k_4\) for protocells under osmotic stress, \(k_6\) for cognitive
systems under overload, or \(k_8\) for civilisations under structural
pressure.  
The rest of the paper and the appendices detail these analogies and
show how they can be turned into concrete predictions.
