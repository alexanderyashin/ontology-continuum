% disciplines.tex
% Core v2.5 — Disciplinary Instantiations of the Continuum Framework

\section{Disciplinary Instantiations}
\label{sec:disciplines}

This section outlines how the Continuum Core applies to major scientific
disciplines.  
Each field corresponds to a specific level \(K_x\) and inherits the universal
template
\[
  K = (\Omega, A, P, \Theta, J, C, k)
\]
with domain-appropriate instantiations of state space, axes, potentials,
thresholds, flows, cycles, and continuity.

\subsection{Physics: Levels \texorpdfstring{$K_0$}{K0}–\texorpdfstring{$K_2$}{K2}}
\label{subsec:disciplines-physics}

Physics corresponds to the foundational levels of the hierarchy.

\paragraph{K0 (Metalogical Foundations).}
Sets logical constraints for physical possibility.

\paragraph{K1 (Lattice Physics).}
Discrete state spaces, energy functionals, minimal cycles.
Continuumness:
\[
  k_1 = \frac{|C_{\max}|}{|X|}.
\]

\paragraph{K2 (Percolation / Early Statistical Mechanics).}
Emergent clusters, correlation lengths, minimal percolation cycle.
Continuumness:
\[
  k_2 = P_\infty.
\]

These levels capture the minimal structural conditions for physical
connectivity, correlation, and time.

\subsection{Chemistry: Levels \texorpdfstring{$K_2$}{K2}–\texorpdfstring{$K_4$}{K4}}
\label{subsec:disciplines-chemistry}

Chemistry emerges when connectivity yields stable reaction networks.

\paragraph{K2.}
Statistical mechanics and local fluctuations set the background.

\paragraph{K3 (Autocatalytic Sets).}
Reaction axes \(A_{\mathrm{react}}\), chemical potentials,
feasibility thresholds \(\Theta_{\mathrm{react}}\),
reaction flows \(J_{\mathrm{chem}}\),
RAF cycles.

\paragraph{K4 (Protocells).}
Membrane axes (inside/outside, permeability),
osmotic and electrochemical potentials,
membrane thresholds \(\Theta_{\mathrm{mem}}\),
transport flows, metabolic cycles.

Chemistry is the interplay of fluxes, thresholds, and cycles that stabilise
compartmentalised reaction systems.

\subsection{Biology: Levels \texorpdfstring{$K_4$}{K4}–\texorpdfstring{$K_6$}{K6}}
\label{subsec:disciplines-biology}

Biology emerges when protocellular systems develop excitation,
information flow, and stable cognitive models.

\paragraph{K4.}
Compartmentalisation, gradients, metabolism.

\paragraph{K5 (Neuronal Systems).}
Excitable membranes with thresholds \(\Theta_{\mathrm{spike}}\),
ion flows,
spike cycles,
neuronal connectivity.

\paragraph{K6 (Cognitive Systems).}
Model axes \(A^{(6)}\),
prediction-error potentials,
cognitive thresholds \(\Theta_{\mathrm{cog}}\),
information flows,
cognitive cycles and attractors.

Biology is the domain where structural tension, cycles, and flows produce
adaptation, agency, and model-based behaviour.

\subsection{Sociology: Level \texorpdfstring{$K_7$}{K7}}
\label{subsec:disciplines-sociology}

Social systems arise when cognitive agents produce stable interaction
patterns constrained by norms and roles.

\paragraph{State Space.}
Agents, roles, communications.

\paragraph{Axes.}
Social-difference axes, role axes, normative axes.

\paragraph{Potentials.}
Cooperation, trust, conflict potentials.

\paragraph{Thresholds.}
Cooperation threshold \(\Theta_{\mathrm{coop}}\).

\paragraph{Flows.}
Communication flows \(J_{\mathrm{comm}}\).

\paragraph{Cycles.}
Institutional cycles \(C_{\mathrm{inst}}\).

\paragraph{Continuumness.}
\[
  k_7 = f(\text{communication coherence},\ \text{institutional persistence}).
\]

\subsection{Civilizational Theory: Level \texorpdfstring{$K_8$}{K8}}
\label{subsec:disciplines-civilization}

Civilizational systems involve long-range structures such as legal codes,
technologies, infrastructures, and symbolic systems.

\paragraph{State Space.}
Technologies, institutions, infrastructures, symbols.

\paragraph{Axes.}
Technological, legal, economic, symbolic.

\paragraph{Potentials.}
Stability, entropy, innovation potentials.

\paragraph{Thresholds.}
Civilizational stability \(\Theta_{\mathrm{stab}}\).

\paragraph{Flows.}
Resource, information, and institutional flows.

\paragraph{Cycles.}
Civilizational stability cycles \(C_{\mathrm{stab}}\).

\paragraph{Continuumness.}
\[
  k_8 = f(\text{infrastructure stability},\ \text{cycle stability}).
\]

\subsection{Knowledge Theory and Metatheory: Levels \texorpdfstring{$K_9$}{K9}–\texorpdfstring{$K_{12}$}{K12}}
\label{subsec:disciplines-metatheory}

The final levels correspond to theories about theories, models about models,
and integration of semantic structures.

\paragraph{K9 (Metatheoretical Systems).}
Theories, paradigms, and scientific cycles.

\paragraph{K10 (Metamodels).}
Categories of models and functorial relations.

\paragraph{K11 (Meta-Semantics).}
Semantic spaces and interpretational consistency.

\paragraph{K12 (Meta-Integration).}
Integration of semantic and metamodel levels.

These levels handle coherence conditions at the boundary of expressibility
and global structural integration.

\subsection{Compact Overview}
\label{subsec:disciplines-summary}

The Continuum Framework unifies the disciplines through a single structure:
\[
  (\Omega, A, P, \Theta, J, C, k),
\]
with each field arising from a different instantiation of these components.
All disciplines inherit the same dynamics, thresholds, and criteria for
emergence, growth, and collapse, ensuring cross-domain consistency.
