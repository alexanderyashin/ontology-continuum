% appendix_levels.tex
% Core v2.5 — Appendix E: Level Definitions K0–K12

\section{Appendix E: Level Definitions \texorpdfstring{$K_0$--$K_{12}$}{K0--K12}}
\label{sec:appendix-levels}

This appendix collects the compact formal definitions of all continua
\(
  K_0, K_1, \dots, K_{12}
\)
as instances of the universal template
\[
  K = \bigl( \Omega, A, P, \Theta, J, C, k \bigr)
\]
with an associated state space \(S(K)\).
Each level \(K_x\) is specified by its state space, axes, potentials,
thresholds, flows, cycles and the corresponding continuumness.

For brevity we write \(\Omega_x = \Omega(K_x)\), \(A_x = A(K_x)\),
\(P_x = P(K_x)\), \(\Theta_x = \Theta(K_x)\), \(J_x = J(K_x)\),
\(C_x = C(K_x)\), and \(k_x = k(K_x)\).

\subsection{\texorpdfstring{$K_0$}{K0}: Metalogical Layer}
\label{subsec:appendix-levels-k0}

\paragraph{State space.}
An abstract metalogical domain
\[
  S(K_0) = \mathcal{P} \times \mathcal{L},
\]
where \(\mathcal{P}\) is the space of conditions of possibility and
\(\mathcal{L}\) the space of metalaws.

\paragraph{Axes.}
No internal axes in the usual sense:
\[
  A_0 = \varnothing.
\]
Differences are defined meta-level as logical distinctions inside
\(\mathcal{P}\) and \(\mathcal{L}\).

\paragraph{Potentials and thresholds.}
Potentials encode logical consistency, thresholds represent
consistency bounds (e.g.\ non-contradiction).

\paragraph{Flows and cycles.}
Flows are inference operations; cycles are consistency-preserving
reasoning loops. Continuumness is trivial:
\[
  k_0 =
    \begin{cases}
      1, & \text{if } \Omega_0 \neq \varnothing,\\
      0, & \text{otherwise.}
    \end{cases}
\]

\subsection{\texorpdfstring{$K_1$}{K1}: Lattice Continuum}
\label{subsec:appendix-levels-k1}

\paragraph{State space.}
Let \(X\) be a finite lattice (graph) with sites \(i\).
States are assignments of local variables:
\[
  S(K_1) = \{\, s : X \to V \,\},
\]
where \(V\) is a finite set (e.g.\ occupancy or spin values).

\paragraph{Axes.}
A single axis \(A_1\) describing local state (e.g.\ occupied/vacant
or spin up/down).

\paragraph{Potentials.}
Energy functional \(E(s)\), local fields, coupling strengths.

\paragraph{Thresholds.}
Connectivity threshold \(\Theta_1\) determining the formation of a
macroscopic cluster.

\paragraph{Flows and cycles.}
Neighbour interactions define flows; a trivial cycle guarantees
time consistency. Continuumness:
\[
  k_1(p) = \frac{\lvert C_{\max}(p) \rvert}{\lvert X \rvert},
\]
where \(C_{\max}(p)\) is the largest connected cluster at occupancy \(p\).

\subsection{\texorpdfstring{$K_2$}{K2}: Percolation Continuum}
\label{subsec:appendix-levels-k2}

\paragraph{State space.}
Infinite lattice or thermodynamic limit of \(K_1\):
\[
  S(K_2) = \lim_{\lvert X \rvert\to\infty} S(K_1).
\]

\paragraph{Axes.}
Occupancy axis \(A_1\) and cluster-connectivity axis \(A_2\)
(e.g.\ finite vs.\ infinite cluster membership).

\paragraph{Potentials.}
Bond/site probabilities; correlation-related quantities.

\paragraph{Thresholds.}
Percolation threshold
\(
  \Theta_2 = p_c
\),
critical value separating non-percolating and percolating phases.

\paragraph{Flows and cycles.}
Cluster growth/merger as flows.
Minimal cycle period \(\tau(K_2)\) related to correlation time.
Continuumness:
\[
  k_2(p) = P_{\infty}(p),
\]
probability that a randomly chosen site is in the infinite cluster.

\subsection{\texorpdfstring{$K_3$}{K3}: Autocatalytic Reaction Networks}
\label{subsec:appendix-levels-k3}

\paragraph{State space.}
States combine concentrations and reaction network structure:
\[
  S(K_3) =
    \bigl\{ (x, G_{\mathrm{react}}) \bigr\},
\]
where \(x\) is a vector of concentrations and
\(G_{\mathrm{react}}\) is a reaction graph.

\paragraph{Axes.}
Chemical composition axes, reaction-activity axes.

\paragraph{Potentials.}
Chemical potentials, free energies, activities.

\paragraph{Thresholds.}
Reaction feasibility thresholds \(\Theta_{\mathrm{react}}\)
(e.g.\ \(\Delta G \le 0\) and rate thresholds).

\paragraph{Flows and cycles.}
Reaction fluxes as flows; RAF-cycles as cycles.
Continuumness \(k_3\) measures the fraction of the network embedded
in the largest self-sustaining autocatalytic set.

\subsection{\texorpdfstring{$K_4$}{K4}: Protocellular Continuum}
\label{subsec:appendix-levels-k4}

\paragraph{State space.}
Internal and external compositions plus membrane state:
\[
  S(K_4) =
    \bigl\{ (x_{\mathrm{in}}, x_{\mathrm{out}}, m) \bigr\},
\]
where \(m\) represents membrane configuration.

\paragraph{Axes.}
Inside/outside axis, permeability axes, gradient axes
(concentration, osmotic, electrochemical).

\paragraph{Potentials.}
Osmotic pressure, electrochemical potentials, metabolic energy.

\paragraph{Thresholds.}
Membrane stability and transport thresholds
\(\Theta_{\mathrm{mem}}\)
(e.g.\ maximum tension before rupture).

\paragraph{Flows and cycles.}
Transport across the membrane and internal metabolic fluxes.
Metabolic cycles as \(C_4\).
Continuumness \(k_4\) grows with stable gradients and persistent
metabolic cycles.

\subsection{\texorpdfstring{$K_5$}{K5}: Neuronal Continuum}
\label{subsec:appendix-levels-k5}

\paragraph{State space.}
Membrane potentials, channel states, synaptic weights:
\[
  S(K_5) = \bigl\{ (V_i, s_i, w_{ij}) \bigr\}.
\]

\paragraph{Axes.}
Excitability axis, ion-channel axes, synaptic axes.

\paragraph{Potentials.}
Electrical potentials, ion gradients, synaptic drive.

\paragraph{Thresholds.}
Spike threshold \(\Theta_{\mathrm{spike}} = V_{\mathrm{thr}}\),
refractory thresholds, stability conditions for excitability.

\paragraph{Flows and cycles.}
Ion currents, neurotransmitter flows and network activity.
Cycles \(C_5\) are spike cycles and rhythmic network patterns.
Continuumness \(k_5\) combines the fraction of neurons participating in
cycles and their effectiveness.

\subsection{\texorpdfstring{$K_6$}{K6}: Cognitive Continuum}
\label{subsec:appendix-levels-k6}

\paragraph{State space.}
Set of internal models and belief states:
\[
  S(K_6) = \mathcal{M} \times \mathcal{B},
\]
where \(\mathcal{M}\) is the model set and \(\mathcal{B}\) encodes
belief weights or probability distributions.

\paragraph{Axes.}
Model axes \(A^{(6)}_k\), discrepancy axes for prediction error,
valuation and salience axes.

\paragraph{Potentials.}
Prediction-error potentials, value/valence potentials, salience
potentials.

\paragraph{Thresholds.}
Cognitive thresholds \(\Theta_{\mathrm{cog}}\),
in particular prediction-error thresholds \(\Theta_{\mathrm{PE}}\)
at which models must change.

\paragraph{Flows and cycles.}
Information flows, belief updates, model-selection dynamics.
Cycles \(C_6\) are attractor cycles in model space.
Continuumness:
\[
  k_6 =
    H(\Omega_6)\,
    \frac{\lvert C^{(6)}_{\max} \rvert}{\lvert \mathcal{M} \rvert}\,
    \frac{r}{r_{\max}}\,
    \biggl( 1 - \frac{D}{\Theta_{\mathrm{cog,max}}} \biggr)_{+},
\]
where \(r\) is the rank (number) of active axes and
\(D\) is a global discrepancy.

\subsection{\texorpdfstring{$K_7$}{K7}: Social Systems}
\label{subsec:appendix-levels-k7}

\paragraph{State space.}
Agents, roles and communication structures:
\[
  S(K_7) =
    \bigl\{ (A_{\mathrm{agents}},
            R_{\mathrm{roles}},
            G_{\mathrm{comm}}) \bigr\}.
\]

\paragraph{Axes.}
Social-difference axes, role axes, normative axes.

\paragraph{Potentials.}
Cooperation potentials, trust/conflict potentials, legitimacy.

\paragraph{Thresholds.}
Cooperation threshold \(\Theta_{\mathrm{coop}}\),
norm-violation thresholds, stability thresholds for institutions.

\paragraph{Flows and cycles.}
Communication flows \(J_{\mathrm{comm}}\),
resource and influence flows.
Institutional cycles \(C_{\mathrm{inst}}\) as \(C_7\).
Continuumness \(k_7\) measures communication coherence and
institutional persistence.

\subsection{\texorpdfstring{$K_8$}{K8}: Civilizational Systems}
\label{subsec:appendix-levels-k8}

\paragraph{State space.}
Technologies, infrastructures, institutions, symbolic systems:
\[
  S(K_8) =
    \bigl\{ (T_{\mathrm{tech}}, I_{\mathrm{inst}},
            F_{\mathrm{infra}}, S_{\mathrm{sym}}) \bigr\}.
\]

\paragraph{Axes.}
Technological axes, economic axes, legal axes, symbolic axes.

\paragraph{Potentials.}
Stability, entropy, innovation, resource and energy potentials.

\paragraph{Thresholds.}
Civilizational stability threshold \(\Theta_{\mathrm{stab}}\),
infrastructure capacity thresholds, critical-load thresholds.

\paragraph{Flows and cycles.}
Resource flows, information flows, institutional flows.
Civilizational cycles \(C_{\mathrm{stab}}\) (e.g.\ long waves,
rise-and-fall dynamics).
Continuumness \(k_8\) encodes stability of core infrastructures
and persistence of stabilising cycles.

\subsection{\texorpdfstring{$K_9$}{K9}: Metatheoretical Continuum}
\label{subsec:appendix-levels-k9}

\paragraph{State space.}
Theories, paradigms, logics and models:
\[
  S(K_9) = \mathcal{T} \times \mathcal{L}_{\mathrm{logic}},
\]
with \(\mathcal{T}\) the set of theoretical systems.

\paragraph{Axes.}
Logical, ontological, methodological, interpretational axes.

\paragraph{Potentials.}
Explanatory power, coherence, parsimony, empirical adequacy.

\paragraph{Thresholds.}
Logical-contradiction thresholds, empirical thresholds,
heuristic-value thresholds.

\paragraph{Flows and cycles.}
Arguments, proofs, reinterpretations and paradigm shifts.
Cycles \(C_9\) are scientific cycles (normal science and revolutions).
Continuumness \(k_9\) increases with coherence and empirical support.

\subsection{\texorpdfstring{$K_{10}$}{K10}: Metamodel Continuum}
\label{subsec:appendix-levels-k10}

\paragraph{State space.}
Categories of models and functors:
\[
  S(K_{10}) = \mathsf{Cat}_{\mathrm{models}},
\]
the category (or higher structure) whose objects are model categories
and whose morphisms are functors.

\paragraph{Axes.}
Functorial axes, structural axes, abstraction level axes.

\paragraph{Potentials.}
Coherence, compositionality and interpretability of metamodels.

\paragraph{Thresholds.}
Coherence thresholds (compatibility of functors and compositions).

\paragraph{Flows and cycles.}
Functorial flows, adjunctions and equivalences.
Cycles \(C_{10}\) are closed functorial chains.
Continuumness \(k_{10}\) measures categorical coherence.

\subsection{\texorpdfstring{$K_{11}$}{K11}: Meta-Semantic Continuum}
\label{subsec:appendix-levels-k11}

\paragraph{State space.}
Semantic spaces of metamodels:
\[
  S(K_{11}) = \mathcal{S}_{\mathrm{meta}},
\]
encoding meaning assignments to model structures.

\paragraph{Axes.}
Semantic axes, interpretational axes, reference axes.

\paragraph{Potentials.}
Semantic coherence, expressivity, stability of meaning.

\paragraph{Thresholds.}
Semantic stability thresholds \(\Theta_{\mathrm{sem}}\),
ambiguity and inconsistency thresholds.

\paragraph{Flows and cycles.}
Interpretation flows, translation between representations.
Cycles \(C_{11}\) are meta-semantic stabilisation loops.
Continuumness \(k_{11}\) grows with semantic coherence and
resistance to drift.

\subsection{\texorpdfstring{$K_{12}$}{K12}: Meta-Integrative Continuum}
\label{subsec:appendix-levels-k12}

\paragraph{State space.}
Integration space of all semantic and metatheoretical levels:
\[
  S(K_{12}) = \mathcal{I}_{\mathrm{global}},
\]
collecting integrative structures over \(K_0,\dots,K_{11}\).

\paragraph{Axes.}
Integrative axes across metatheories, cross-level mapping axes.

\paragraph{Potentials.}
Global coherence, integrability, cross-domain consistency.

\paragraph{Thresholds.}
Global integrability thresholds (limits beyond which
no consistent integration is possible).

\paragraph{Flows and cycles.}
Cross-level integrative flows, reflexive feedback loops
between lower and meta-levels.
Cycles \(C_{12}\) are global integrative cycles.
Continuumness \(k_{12}\) captures the degree of global structural
consistency across the entire hierarchy.

\subsection{Compact Summary}
\label{subsec:appendix-levels-summary}

Each level \(K_x\) is a concrete instantiation of
\(
  (\Omega, A, P, \Theta, J, C, k)
\)
on a specific state space \(S(K_x)\), obeying:

\begin{itemize}
  \item existence iff \(\Omega_x \neq \varnothing\) and \(C_x \neq \varnothing\),
  \item monotonic dimension growth while \(k_x > 0\),
  \item compatibility with metaspace:
        \(\Omega_x \subseteq \Omega(M_x)\),
  \item collapse iff \(\Omega_x = \varnothing\) or \(C_x = \varnothing\).
\end{itemize}

This appendix provides the cross-level reference that links the compact core,
the hierarchy and the continuity conditions into a single coherent structure.
