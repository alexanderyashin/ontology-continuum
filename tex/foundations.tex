% foundations.tex
% Core v2.5 — Formal Foundations of the Continuum Framework

\section{Foundations of the Continuum Framework}
\label{sec:foundations}

This section introduces the formal foundations of the Continuum Framework.
We fix the basic notation, define continua as structured dynamical systems,
specify the admissible state set \(\Omega(K)\) and its boundary
\(\partial\Omega(K)\), and state the core axioms and meta-theorems
governing all levels \(K_0,\dots,K_{12}\).

\subsection{Notation and Basic Objects}
\label{subsec:foundations-notation}

Throughout this work we use the following conventions.

\begin{itemize}
  \item A \emph{continuum} is denoted by \(K\) with an optional level index,
        e.g.\ \(K_x\) for \(x=0,\dots,12\).
  \item The \emph{state space} of \(K\) is a set \(S(K)\).
        The subset of admissible states is denoted by \(\Omega(K) \subseteq S(K)\).
  \item The \emph{axes of difference} of \(K\) form a set
        \(A(K) = \{A_j\}_{j \in J_A}\).
  \item The \emph{potentials} of \(K\) form a set
        \(P(K) = \{P_i\}_{i \in J_P}\).
  \item The \emph{thresholds} of \(K\) form a set
        \(\Theta(K) = \{\Theta_\ell\}_{\ell \in J_\Theta}\).
  \item The \emph{flows} of \(K\) form a set
        \(J(K) = \{J_m\}_{m \in J_J}\).
  \item The \emph{cycles} of \(K\) form a set
        \(C(K) = \{C_n\}_{n \in J_C}\).
  \item The \emph{continuumness} (or \emph{degree of connectivity})
        is a function
        \[
          k(K) : \Omega(K) \to [0,1].
        \]
\end{itemize}

Whenever the meaning is clear, we omit the explicit dependence on \(K\)
and write \(\Omega, A, P, \Theta, J, C, k\).

\subsection{Continuum as a Structured System}
\label{subsec:foundations-continuum}

\paragraph{Definition (Continuum).}
A \emph{continuum} is a tuple
\[
  K = \bigl(\Omega, A, P, \Theta, J, C, k\bigr)
\]
together with an ambient state space \(S(K)\) such that:
\begin{enumerate}
  \item \textbf{Axes of difference.}
        Each axis \(A_j\) is a function
        \[
          A_j : S(K) \to \mathcal{A}_{A_j},
        \]
        where \(\mathcal{A}_{A_j}\) is the set of admissible values
        of that axis.
  \item \textbf{Potentials.}
        Each potential \(P_i\) is a function
        \[
          P_i : S(K) \to \mathcal{D}_{P_i},
        \]
        where \(\mathcal{D}_{P_i}\) is a domain of admissible values
        (for example intervals in \(\mathbb{R}\) or subsets of \(\mathbb{R}^n\)).
  \item \textbf{Thresholds.}
        Each threshold \(\Theta_\ell\) is a function (or parameter)
        specifying a boundary between qualitatively different regimes.
        We write collectively
        \(\vec{\Theta} = (\Theta_\ell)_{\ell \in J_\Theta}\).
  \item \textbf{Flows.}
        Each flow \(J_m\) is a mapping describing a transport or
        transformation along the axes,
        for example
        \[
          J_m : S(K) \times \mathbb{R} \to \mathcal{F}_m,
        \]
        where \(t \in \mathbb{R}\) denotes time.
  \item \textbf{Cycles.}
        Each cycle \(C_n\) is a closed trajectory in state space,
        i.e.\ a map \(\gamma_n : [0,\tau_n] \to S(K)\) such that
        \(\gamma_n(0) = \gamma_n(\tau_n)\).
  \item \textbf{Continuumness.}
        The function \(k : \Omega(K) \to [0,1]\) measures the global
        connectedness and viability of the continuum.
\end{enumerate}

Intuitively, a continuum exists whenever there is a non-empty set of
admissible states equipped with axes, potentials, thresholds, flows, and cycles,
such that at least one non-trivial cycle persists and \(k > 0\).

\subsection{Admissible States and the Boundary \texorpdfstring{\(\partial\Omega(K)\)}{dOmega(K)}}
\label{subsec:foundations-omega}

We now give the universal definition of admissible states and their boundary.

\paragraph{Definition (Admissible states).}
For a continuum \(K\) with state space \(S(K)\), the set of admissible states
\(\Omega(K)\) is defined by
\begin{align*}
  \Omega(K) = \bigl\{\, s \in S(K) \ \big|\ &
    P_i(s) \in \mathcal{D}_{P_i} \ \text{for all } i, \\
  & A_j(s) \in \mathcal{A}_{A_j} \ \text{for all } j, \\
  & C(s) \neq \varnothing, \\
  & \vec{\Theta}(s) \geq 0, \\
  & T(s) \leq 0 \,\bigr\},
\end{align*}
where:
\begin{itemize}
  \item \(C(s)\) denotes the set of cycles that remain viable in state \(s\);
  \item \(\vec{\Theta}(s)\) is the vector of thresholds evaluated in state \(s\);
  \item \(T(s)\) is the structural tension of state \(s\), defined such that
        \(T(s) \leq 0\) means that no threshold is violated.
\end{itemize}

\paragraph{Definition (Boundary of admissible states).}
The boundary of the admissible state set is
\[
  \partial\Omega(K) =
  \bigl\{\, s \in S(K) \ \big|\ 
    T(s) = 0 \ \text{or} \
    \exists i:\ P_i(s) \in \partial\mathcal{D}_{P_i}
    \ \text{or} \
    \exists j:\ A_j(s) \in \partial\mathcal{A}_{A_j}
  \,\bigr\}.
\]
States on \(\partial\Omega(K)\) lie on the thresholds of phase transitions,
dimension growth, or collapse of the continuum.

\subsection{Time, Cycles and the Evolution Operator}
\label{subsec:foundations-time}

Time in the Continuum Framework is defined in terms of cycles.

\paragraph{Definition (Minimal cycle period and time scale).}
Let \(C(K)\) be the set of cycles of a continuum \(K\), and let
\(\tau_n > 0\) denote the period of a cycle \(C_n\).
If there exists at least one non-trivial cycle, the characteristic
time scale of \(K\) is defined as
\[
  \tau(K) = \min_{n} \tau_n.
\]
If \(C(K) = \varnothing\), we set \(\tau(K) = +\infty\); in this case
the continuum is considered dead (\(k=0\)).

\paragraph{Definition (Evolution operator).}
The evolution of a continuum is given by an operator
\[
  E : K(t) \mapsto K(t+\Delta t),
\]
which can be decomposed into the dynamics of the components:
\begin{align*}
  \frac{\mathrm{d}A}{\mathrm{d}t} &= F(A, P, \Theta, J, C, t), \\
  \frac{\mathrm{d}P}{\mathrm{d}t} &= G(A, P, \Theta, J, C, t), \\
  \frac{\mathrm{d}\Theta}{\mathrm{d}t} &= H(A, P, \Theta, J, C, t), \\
  \frac{\mathrm{d}J}{\mathrm{d}t} &= Q(A, P, \Theta, J, C, t), \\
  \frac{\mathrm{d}C}{\mathrm{d}t} &= S(A, P, \Theta, J, C, t), \\
  \frac{\mathrm{d}k}{\mathrm{d}t} &= U(A, P, \Theta, J, C, k, t).
\end{align*}
Here \(F, G, H, Q, S, U\) are level-dependent operators, constrained
by the axioms and the structure of the metaspace.

\subsection{Core Axioms}
\label{subsec:foundations-axioms}

We recall three axioms that underlie all levels of the Continuum Framework.

\paragraph{Axiom 0.1 (Difference and Connectivity).}
Metapressure generates non-identity of possible states.
Difference is the minimal form of contradiction.
Connectivity of a continuum is a function of difference: if there are
no differences (\(\Delta = 0\)), connectivity vanishes and
\(k(t) = 0\).
The basic building blocks become elements of a continuum only when
they can be assigned distinguishable coordinates.
A continuum exists if and only if there are at least two distinguishable
states.

\paragraph{Axiom 0.3 (Logical Supremacy of \(K_0\)).}
The level \(K_0\) does not belong to the sets of continua, differences,
or dimensions.
It is given as a pair \((\mathcal{P}, \mathcal{L})\), the space of
conditions of possibility and metalaws.
There is a relation of logical superiority \(K_0 \succ K_i\) if and only if
all structures of \(K_i\) (state sets \(\mathcal{C}_i\), differences
\(\mathcal{D}_i\), and dimensions \(\mathcal{R}_i\)) are contained in
the domain of definition of \(K_0\).
Here ``greater'' refers to logical supremacy: \(K_0\) exceeds any \(K_i\)
in the sense of defining the conditions of possibility of its structures,
not in a physical or metric sense.

\paragraph{Principle of Conjugate Hierarchy \(K \leftrightarrow M\).}
Each continuum \(K_x\) generates a metaspace \(M_{x+1}\) as the space
of admissible states induced by the structure of \(K_x\).
Conversely, \(M_{x+1}\) defines the conditions of possibility for the
existence of \(K_x\) and the potential emergence of \(K_{x+1}\).
The pairs \((K_x, M_{x+1})\) form a conjugate hierarchy
\((K_0, M_1), (K_1, M_2), \dots\), with
\(\mathrm{DoF}(K_x) \leq \mathrm{DoF}(M_{x+1})\) and
\(A_{K_x} \subseteq A_{M_{x+1}}\).

\subsection{Global Theorems}
\label{subsec:foundations-theorems}

In this subsection we state three global theorems that apply to all levels
of the Continuum Framework. Proofs are given in later sections for
particular classes of continua.

\paragraph{Theorem 1 (Monotonicity of Dimension).}
Let \(K(t)\) be a continuum with \(k(t) > 0\) on a time interval
\([t_0, t_1]\), and let \(\dim K(t)\) denote the dimension of \(K(t)\).
Then
\[
  \dim K(t_1) \geq \dim K(t_0).
\]
In particular, while the continuum is alive (\(k>0\)), its dimension
cannot decrease; it can either remain constant or increase.
A decrease of dimension is possible only through the death of the
continuum, i.e.\ \(\Omega(K) = \varnothing\) and \(k = 0\).

\paragraph{Theorem 2 (Death as Loss of Cycles).}
A continuum \(K\) ceases to exist if and only if, for all configurations
of differences and for all possible dimensions, the structural tension
\(T\) exceeds the corresponding threshold \(\Theta\).
In this case there are no reachable states, \(\Omega(K) = \varnothing\),
and the set of cycles becomes empty: \(C(K) = \varnothing\).
Equivalently, death can be characterised by the condition
\[
  k(K) = 0 \quad \Longleftrightarrow \quad C(K) = \varnothing.
\]

\paragraph{Theorem 3 (Compatibility of Continuum and Metaspace).}
For each level \(x\), the continuum \(K_x\) exists if and only if
its admissible state set is contained in the admissible set of its
metaspace:
\[
  K_x \text{ exists} \quad \Longleftrightarrow \quad
  \Omega(K_x) \subseteq \Omega(M_x).
\]
In particular, violations of the constraints of \(M_x\) force \(K_x\)
to leave its admissible region and eventually lead to \(k_x = 0\).
This provides a general criterion of consistency between continua and
their metaspaces.
\medskip

The definitions and principles stated in this section are the formal
foundation of all subsequent levels \(K_0,\dots,K_{12}\).
Each level is obtained by specifying concrete choices for the state
space \(S(K_x)\), axes \(A(K_x)\), potentials \(P(K_x)\), thresholds
\(\Theta(K_x)\), flows \(J(K_x)\), cycles \(C(K_x)\), and the measure
of continuumness \(k(K_x)\), subject to the universal axioms and the
compatibility conditions with the corresponding metaspaces.
