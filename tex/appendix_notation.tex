% appendix_notation.tex
% Core v2.5 — Appendix D: Notation and Symbol Glossary

\section{Appendix D: Notation and Symbol Glossary}
\label{sec:appendix-notation}

This appendix defines all symbols, sets, operators and quantities used across
the Continuum Framework. It serves as a unified reference for notation
throughout all sections and appendices.

\subsection{Sets and Spaces}
\label{subsec:appendix-notation-sets}

\begin{itemize}
  \item \(S(K)\) — state space of continuum \(K\).
  \item \(\Omega(K)\subseteq S(K)\) — admissible region.
  \item \(\partial\Omega(K)\subseteq S(K)\) — boundary of admissibility.
  \item \(M_x\) — metaspace corresponding to continuum \(K_x\).
  \item \(\mathcal{A}_{A_j}\) — domain of axis \(A_j\).
  \item \(\mathcal{D}_{P_i}\) — domain of potential \(P_i\).
  \item \(\mathcal{D}_{\pi_i}\) — domain of parameter \(\pi_i\).
  \item \(J_A,J_P,J_\Theta,J_J,J_C\) — index sets for axes, potentials,
        thresholds, flows and cycles.
\end{itemize}

\subsection{Core Components}
\label{subsec:appendix-notation-core}

\begin{itemize}
  \item \(A(K)=\{A_j\}\) — set of axes of difference.
  \item \(P(K)=\{P_i\}\) — set of potentials.
  \item \(\Theta(K)=\{\Theta_\ell\}\) — threshold family.
  \item \(J(K)=\{J_m\}\) — flows.
  \item \(C(K)=\{C_n\}\) — cycles (closed trajectories).
  \item \(k(K):\Omega(K)\to[0,1]\) — continuumness (viability).
  \item \(T(s)\) — structural tension at state \(s\).
\end{itemize}

\subsection{Operators}
\label{subsec:appendix-notation-operators}

\begin{itemize}
  \item \(E\) — universal evolution operator.
  \item \(\Psi_{x\rightarrow x+1}\) — transition operator between continua.
  \item \(F,G,H,Q,S,R,U\) — component evolution operators:
    \begin{itemize}
      \item \(F\): axis dynamics,
      \item \(G\): potential dynamics,
      \item \(H\): threshold dynamics,
      \item \(Q\): tension dynamics,
      \item \(S\): flow dynamics,
      \item \(R\): cycle dynamics,
      \item \(U\): continuumness dynamics.
    \end{itemize}
  \item \(W\) — parameter evolution operator.
\end{itemize}

\subsection{Dynamics}
\label{subsec:appendix-notation-dynamics}

Differential relations obey the unified scheme:
\begin{align*}
  \frac{dA}{dt} &= F(A,P,\Theta,J,C,t), \\
  \frac{dP}{dt} &= G(A,P,\Theta,J,C,t), \\
  \frac{d\Theta}{dt} &= H(A,P,\Theta,J,C,t), \\
  \frac{dT}{dt} &= Q(A,P,\Theta,J,C,T,t), \\
  \frac{dJ}{dt} &= S(A,P,\Theta,J,C,t), \\
  \frac{dC}{dt} &= R(A,P,\Theta,J,C,T,t), \\
  \frac{dk}{dt} &= U(A,P,\Theta,J,C,T,k,t), \\
  \frac{d\Pi}{dt} &= W(A,P,\Theta,J,C,T,k,\Pi,t).
\end{align*}

\subsection{Temporal Quantities}
\label{subsec:appendix-notation-time}

\begin{itemize}
  \item \(\tau_n\) — period of cycle \(C_n\).
  \item \(\tau(K)=\min_n \tau_n\) — minimal active cycle period.
\end{itemize}

\subsection{Dimensions}
\label{subsec:appendix-notation-dim}

\begin{itemize}
  \item \(\dim K = |A(K)|\) — dimension of the continuum.
  \item \(\Theta_{\mathrm{dim}}\) — dimensional-threshold value.
  \item \(\mathrm{DoF}(K)\) — degrees of freedom of \(K\).
\end{itemize}

\subsection{Threshold Structures}
\label{subsec:appendix-notation-thresholds}

\begin{itemize}
  \item \(\Theta_{\mathrm{stab}}\) — stability thresholds.
  \item \(\Theta_{\mathrm{react}}\) — reaction thresholds.
  \item \(\Theta_{\mathrm{mem}}\) — membrane/mechanical thresholds.
  \item \(\Theta_{\mathrm{spike}}\) — neuronal spike thresholds.
  \item \(\Theta_{\mathrm{cog}}\) — cognitive thresholds.
  \item \(\Theta_{\mathrm{coop}}\) — cooperation thresholds.
  \item \(\Theta_{\mathrm{sem}}\) — semantic thresholds.
  \item \(\Theta_{\mathrm{cycle}}\) — cycle viability thresholds.
\end{itemize}

\subsection{Tension}
\label{subsec:appendix-notation-tension}

\[
  T : S(K)\to\mathbb{R}
\]
represents structural tension.  
A state is admissible iff:
\[
  T(s)\le 0.
\]
Boundary states satisfy:
\[
  T(s)=0.
\]

\subsection{Parameter Sets}
\label{subsec:appendix-notation-params}

\[
  \Pi(K)=\{\pi_1,\dots,\pi_n\},
\]
all free, adaptive and structural parameters of the continuum.

\subsection{Metaspace Conditions}
\label{subsec:appendix-notation-meta}

Compatibility with the metaspace requires:
\[
  \Omega(K_x)\subseteq\Omega(M_x), \qquad
  A(K_x)\subseteq A(M_x).
\]
Degrees of freedom obey:
\[
  \mathrm{DoF}(K_x) \le \mathrm{DoF}(M_{x+1}).
\]

\subsection{Glossary of Frequently Used Symbols}
\label{subsec:appendix-notation-glossary}

\begin{itemize}
  \item \(s\) — state of a continuum.
  \item \(\partial\Omega\) — boundary of admissibility.
  \item \(L(s)=T(s)-\Theta(s)\) — landscape function.
  \item \(p_c\) — percolation threshold.
  \item \(V_{\mathrm{thr}}\) — neuronal spike threshold.
  \item \(D\) — global cognitive discrepancy.
  \item \(T_{\mathrm{comm}}\) — communication tension.
  \item \(T_{\mathrm{inst}}\) — institutional tension.
\end{itemize}

\subsection{Compact Summary}
\label{subsec:appendix-notation-summary}

This appendix provides a unified glossary for all components, operators, 
thresholds, tensions and domains across continua \(K_0\)–\(K_{12}\).  
It ensures cross-level consistency and a single coherent symbolic system.
