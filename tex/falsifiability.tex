% falsifiability.tex
% Core v2.5 — Falsifiability and Empirical Tests

\section{Falsifiability and Empirical Tests}
\label{sec:falsifiability}

This section provides the universal and level-specific criteria under which
the Continuum Framework can be empirically or mathematically falsified.
Falsifiability is formulated directly in terms of the core components
\[
  (\Omega, A, P, \Theta, J, C, k).
\]

\subsection{Universal Falsifiability Criteria}
\label{subsec:falsifiability-universal}

The framework is falsifiable if any of the following universal conditions
are violated by observation, experiment, or mathematical proof.

\paragraph{Criterion F1 (Admissible Region Failure).}
There must exist non-empty admissible states:
\[
  \Omega(K) \neq \varnothing.
\]
If a continuum is claimed to exist but no admissible states exist under its
own constraints, the model is falsified.

\paragraph{Criterion F2 (Cycle Requirement).}
Continuumness is strictly tied to the existence of cycles:
\[
  k(K) > 0 \iff C(K) \neq \varnothing.
\]
If a living continuum is observed without any persistent cycle,
the theory is false.

\paragraph{Criterion F3 (Threshold Structure).}
Thresholds must separate admissible and inadmissible states:
\[
  T(s)\le0 \ \text{inside }\Omega,\qquad
  T(s)=0 \ \text{on }\partial\Omega.
\]
If thresholds do not regulate transitions as predicted, the model is falsified.

\paragraph{Criterion F4 (Monotonic Dimension).}
If a continuum remains alive (\(k>0\)) and its dimension decreases,
the theory is falsified.

\paragraph{Criterion F5 (Metaspace Compatibility).}
The existence condition
\[
  \Omega(K_x) \subseteq \Omega(M_x)
\]
must hold for every level.
Violation falsifies the framework.

\subsection{Level-Specific Tests}
\label{subsec:falsifiability-levels}

We summarise concrete falsification tests for each major level.

\paragraph{Physics (K1–K2).}
\begin{itemize}
  \item Violation of percolation threshold behaviour.
  \item Existence of connectivity without cycles in K1.
  \item Observation of time without any periodic process.
\end{itemize}

\paragraph{Chemistry (K3–K4).}
\begin{itemize}
  \item Autocatalytic networks persist with no cycles.
  \item Protocells maintain gradients without membrane thresholds.
  \item Transport occurs across membranes without crossing \(\Theta_{\mathrm{mem}}\).
\end{itemize}

\paragraph{Biology (K4–K6).}
\begin{itemize}
  \item Neurons spike without surpassing \(\Theta_{\mathrm{spike}}\).
  \item Cognitive systems stabilise models without cycles in their attractor space.
  \item Prediction-error dynamics violate the defined potentials.
\end{itemize}

\paragraph{Social Systems (K7).}
\begin{itemize}
  \item Stable institutions without communication cycles.
  \item Cooperation emerging below \(\Theta_{\mathrm{coop}}\).
  \item Social collapse occurring with all thresholds satisfied.
\end{itemize}

\paragraph{Civilizations (K8).}
\begin{itemize}
  \item Civilizational collapse without violation of \(\Theta_{\mathrm{stab}}\).
  \item Long-term stability without cycles \(C_{\mathrm{stab}}\).
\end{itemize}

\paragraph{Knowledge and Metatheory (K9–K12).}
\begin{itemize}
  \item Logical inconsistency in sustained paradigms (K9).
  \item Functorial incoherence in persistent metamodels (K10).
  \item Semantic instability despite all thresholds being satisfied (K11).
  \item Global integrative collapse without threshold violation (K12).
\end{itemize}

\subsection{Experimental and Observational Protocols}
\label{subsec:falsifiability-protocols}

The framework proposes domain-specific tests:

\paragraph{K2 Tests.}
Percolation experiments on lattices; measurement of \(P_\infty\).

\paragraph{K3 Tests.}
Sustained autocatalytic cycles under perturbation; reaction feasibility.

\paragraph{K4 Tests.}
Membrane osmotic collapse; transport thresholds; metabolic cycle persistence.

\paragraph{K5 Tests.}
Patch-clamp excitation threshold; spike cycle deformation; refractory thresholds.

\paragraph{K6 Tests.}
Cognitive inconsistency measurement; attractor stability; prediction-error thresholds.

\paragraph{K7 Tests.}
Institutional cycle duration; coherence of communication flows.

\paragraph{K8 Tests.}
Stability metrics of infrastructures; analysis of long-term civilizational cycles.

\paragraph{K9–K12 Tests.}
Logical coherence, structural mapping, semantic consistency.

\subsection{Meta-Falsifiability}
\label{subsec:falsifiability-meta}

The framework also predicts that:

\begin{itemize}
  \item Any model \(M\) representing a continuum must itself be a continuum
    under \(K_9\) or \(K_{10}\).
  \item Falsification of a model requires detecting violation of its own
    \((\Omega, A, P, \Theta, J, C, k)\).
  \item Meta-falsification occurs when an entire class of models violates
    metaspace constraints.
\end{itemize}

\subsection{Compact Summary}
\label{subsec:falsifiability-summary}

Falsifiability is encoded directly in the structure of continua:
\[
  K \text{ is false} \iff
  \bigl(
    \Omega=\varnothing \ \lor \
    C=\varnothing\text{ while }k>0 \ \lor \
    \dim\downarrow\text{ while }k>0 \ \lor \
    \Omega(K_x)\not\subseteq\Omega(M_x)
  \bigr).
\]
The framework is thus fully testable across all physical,
chemical, biological, cognitive, social, civilizational, and metatheoretical
domains.
