% falsifiability.tex
% Core v2.5 — Falsifiability and Empirical Tests

\section{Falsifiability and Empirical Tests}
\label{sec:falsifiability}

Falsifiability in the Continuum Framework is expressed entirely in terms of
the structural primitives
\[
  (\Omega, A, P, \Theta, J, C, k).
\]
A claim about a continuum is false if its observed behaviour contradicts
the universal structural constraints linking admissible states,
thresholds, cycles and continuumness.

The following criteria apply across all continua \(K_x\), from physical
levels \(K_1\)–\(K_2\) to metatheoretical levels \(K_9\)–\(K_{12}\).

% -------------------------------------------------------------
\subsection{Universal Falsifiability Criteria}
\label{subsec:falsifiability-universal}

A continuum \(K\) is falsified if \emph{any} of the following conditions are
violated.

\paragraph{F1: Admissible Region Non-Emptiness.}
Every existing continuum must satisfy
\[
  \Omega(K) \neq \varnothing.
\]
If a system is claimed to exist dynamically but its
potentials, thresholds or flows imply
\(
  \Omega(K)=\varnothing,
\)
the claim is false.

\paragraph{F2: Cycle–Continuity Equivalence.}
Continuumness is possible only if cycles exist:
\[
  k(K)>0 \iff C(K)\neq\varnothing.
\]
If a system exhibits long-term coherence without any closed cycle
(metabolic, neural, social, semantic), the framework is falsified.

\paragraph{F3: Threshold–Boundary Correctness.}
Thresholds must define the boundary between admissible and inadmissible
states:
\[
  T(s)\le 0 \iff s\in\Omega(K),
  \qquad
  T(s)=0 \iff s\in\partial\Omega(K).
\]
If transitions occur without crossing \(\Theta\), falsification follows.

\paragraph{F4: Monotonic Dimension Under Survival.}
If a continuum remains alive (\(k>0\)) but its dimension decreases:
\[
  \dim K(t+\Delta t) < \dim K(t),
\]
the theory is falsified.  
Dimension may drop only through collapse (\(k\to0\)).

\paragraph{F5: Metaspace Compatibility.}
Each continuum must obey:
\[
  \Omega(K_x)\subseteq\Omega(M_x),
\]
where \(M_x\) is the metaspace enabling \(K_x\).  
Violation implies structural impossibility.

\paragraph{F6: Boundary Divergence at Thresholds.}
Near thresholds, the minimal cycle period must diverge:
\[
  \tau(K)\to\infty
  \quad\text{as}\quad
  T\to\Theta.
\]
If a continuum approaches its threshold without temporal dilation,
the universal continuity law fails.

% -------------------------------------------------------------
\subsection{Level-Specific Falsifiability Tests}
\label{subsec:falsifiability-levels}

Each level has its own concrete falsification conditions derived from its
state space, axes, potentials and thresholds.

% --- Physics --------------------------------------------------
\paragraph{K1–K2 (Physics).}
\begin{itemize}
  \item Observation of persistent temporal structure without cycles.
  \item Connectivity transition not governed by a threshold \(p_c\).
  \item Existence of a macroscopic phase with no divergence of \(\tau\).
\end{itemize}

% --- Chemistry ------------------------------------------------
\paragraph{K3–K4 (Chemistry).}
\begin{itemize}
  \item Autocatalytic sets persisting without a closed RAF-cycle.
  \item Protocells maintaining gradients while violating membrane thresholds.
  \item Transport across membranes without crossing \(\Theta_{\mathrm{mem}}\).
\end{itemize}

% --- Biology --------------------------------------------------
\paragraph{K4–K6 (Biology).}
\begin{itemize}
  \item Neurons emitting spikes without surpassing \(\Theta_{\mathrm{spike}}\).
  \item Failure of spike-cycle closure while the system remains coherent.
  \item Cognitive systems retaining coherence while \(D>\Theta_{\mathrm{PE}}\).
\end{itemize}

% --- Social ---------------------------------------------------
\paragraph{K7 (Social Systems).}
\begin{itemize}
  \item Stable institutions without a complete institutional cycle.
  \item Cooperation emerging below \(\Theta_{\mathrm{coop}}\).
  \item Social collapse occurring despite all thresholds being satisfied.
\end{itemize}

% --- Civilizational ------------------------------------------
\paragraph{K8 (Civilizational Systems).}
\begin{itemize}
  \item Civilizations collapsing with \(T_8<\Theta_{\mathrm{stab}}\).
  \item Long-term stability without stabilising cycles \(C_{\mathrm{stab}}\).
  \item Absence of hysteresis near structural thresholds.
\end{itemize}

% --- Meta -----------------------------------------------------
\paragraph{K9–K12 (Knowledge, Models, Semantics).}
\begin{itemize}
  \item Logical inconsistency persisting without collapse in \(K_9\).
  \item Functorial incoherence tolerated in \(K_{10}\).
  \item Semantic instability with all \(\Theta_{\mathrm{sem}}\) satisfied.
  \item Global integrative collapse in \(K_{12}\) without threshold violation.
\end{itemize}

% -------------------------------------------------------------
\subsection{Cross-Level Falsifiability}
\label{subsec:falsifiability-cross}

The Continuum Framework predicts that all transitions
\(\Psi_{x\rightarrow x+1}\) must obey:

\begin{itemize}
  \item emergence of a new axis \(A_{x+1}\notin A_x\),
  \item expansion of the admissible set:
        \(\Omega(K_x)\subseteq\Omega(K_{x+1})\),
  \item monotonicity of dimension while \(k>0\),
  \item divergence of cycle periods near the dimensional threshold
        \(\Theta_{\mathrm{dim}}\).
\end{itemize}

If any transition violates these conditions, the entire operator
\(\Psi_{x\rightarrow x+1}\) is falsified.

% -------------------------------------------------------------
\subsection{Experimental and Observational Protocols}
\label{subsec:falsifiability-protocols}

The framework offers domain-specific tests:

\paragraph{K2:} Lattice percolation; measurement of \(P_\infty(p)\).  
\paragraph{K3:} RAF persistence under perturbations.  
\paragraph{K4:} Osmotic rupture curves; membrane-threshold mapping.  
\paragraph{K5:} Patch-clamp measurement of \(\Theta_{\mathrm{spike}}\).  
\paragraph{K6:} Prediction-error saturation experiments.  
\paragraph{K7:} Communication-coherence mapping in institutions.  
\paragraph{K8:} Infrastructure-tension measurement; long-wave analysis.  
\paragraph{K9–K12:} Logical, functorial and semantic consistency tests.

% -------------------------------------------------------------
\subsection{Meta-Falsifiability}
\label{subsec:falsifiability-meta}

Every model that represents a continuum must itself satisfy the
continuum axioms at level \(K_9\) or \(K_{10}\).  
Thus:

\begin{itemize}
  \item A model is falsified if its own \((\Omega, A, P, \Theta, J, C, k)\)
        becomes inconsistent.
  \item A metamodel is falsified if functorial relations break.
  \item A semantic layer is falsified if meaning loses stability
        despite thresholds being respected.
\end{itemize}

This provides a hierarchical falsifiability chain across all levels.

% -------------------------------------------------------------
\subsection{Compact Summary}
\label{subsec:falsifiability-summary}

A continuum is falsified if:
\[
  \Omega=\varnothing
  \quad\lor\quad
  (C=\varnothing\ \text{while}\ k>0)
  \quad\lor\quad
  (\dim\downarrow\ \text{while}\ k>0)
  \quad\lor\quad
  \Omega(K_x)\not\subseteq\Omega(M_x)
  \quad\lor\quad
  \tau\not\to\infty\ \text{as}\ T\to\Theta.
\]

The framework is thus empirically, mathematically and structurally testable
across physics, chemistry, biology, cognition, social systems,
civilizations and metatheoretical domains.
