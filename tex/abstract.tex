This work introduces a unified continuum framework capable of describing
physical, chemical, biological, cognitive, social, civilizational and formal
systems within a single mathematical structure. A continuum is defined as
\(
K=(\Omega,A,P,\Theta,J,C,k),
\)
where admissible states are characterised by threshold-saturated tension,
structural axes, potentials, flows and the existence of nontrivial cycles.
The framework provides universal viability conditions
\(
T(s)\le 0,\ \Omega\neq\varnothing,\ C\neq\varnothing,
\)
together with collapse criteria
\(
k=0
\)
whenever admissible states or cycles vanish. A hierarchy of twelve continua
(K$_0$–K$_{12}$) is presented, ranging from pre-structural physical domains
to recursive ontological systems, with rigorous continuity conditions
(C1–C5) ensuring monotonic growth of structural dimension across levels.
Unified dynamical equations for the operators
\(
(F,G,H,Q,S,U)
\)
govern the evolution of axes, potentials, thresholds, tension, cycles and
continuumness. The framework yields falsifiable predictions for percolation
thresholds, catalytic closure, metabolic and excitability limits, cognitive
prediction windows, coordination and institutional stability, paradigm
coherence and proof-cycle persistence. Falsifiability is established through
five explicit criteria (F1–F5), providing conditions under which empirical
systems refute the model. The result is a compact, coherent and testable
general theory linking structural viability, threshold geometry and
multi-level dynamical organisation.
