% appendix_params.tex
% Core v2.5 — Appendix C: Parameters, Thresholds, and Tension Functions

\section{Appendix C: Parameters, Thresholds, and Tension Functions}
\label{sec:appendix-params}

This appendix defines the parameter sets, threshold functions, and tension
structures used across levels \(K_0\) to \(K_{12}\).  
Parameters include numerical values, functional dependencies, and structural
coefficients that shape the admissible region \(\Omega(K)\) and its boundary
\(\partial\Omega(K)\).

\subsection{Parameter Sets}
\label{subsec:appendix-params-sets}

For each continuum \(K\), the parameter set is:
\[
  \Pi(K) = \{\, \pi_1, \pi_2, \dots, \pi_n \,\},
\]
with each parameter \(\pi_i\) belonging to an admissible domain
\(\mathcal{D}_{\pi_i}\).

Examples include:
\begin{itemize}
  \item physical parameters (bond probability \(p\), coupling strengths),
  \item chemical parameters (rate constants, activities),
  \item biological parameters (channel densities, conductances),
  \item cognitive parameters (prediction-error bounds),
  \item social parameters (communication rates),
  \item civilizational parameters (resource throughput),
  \item metatheoretical parameters (coherence coefficients).
\end{itemize}

\subsection{Threshold Functions}
\label{subsec:appendix-params-thresholds}

Thresholds regulate transitions and define the admissible surface:
\[
  \Theta : S(K) \to \mathbb{R}.
\]

Each threshold may depend on parameters:
\[
  \Theta(s) = \Theta(s;\Pi).
\]

Key threshold classes include:
\begin{itemize}
  \item \emph{dimensional thresholds} \(\Theta_{\mathrm{dim}}\),
  \item \emph{stability thresholds} \(\Theta_{\mathrm{stab}}\),
  \item \emph{reactivity thresholds} \(\Theta_{\mathrm{react}}\),
  \item \emph{membrane thresholds} \(\Theta_{\mathrm{mem}}\),
  \item \emph{excitation thresholds} \(\Theta_{\mathrm{spike}}\),
  \item \emph{cognitive thresholds} \(\Theta_{\mathrm{cog}}\),
  \item \emph{cooperation thresholds} \(\Theta_{\mathrm{coop}}\),
  \item \emph{semantic thresholds} \(\Theta_{\mathrm{sem}}\).
\end{itemize}

\subsection{Threshold Surfaces}
\label{subsec:appendix-params-surfaces}

The admissible boundary is defined by:
\[
  T(s) = \Theta(s;\Pi)
  \qquad \iff \qquad
  s\in \partial\Omega(K).
\]

The structure may be:
\begin{itemize}
  \item smooth,
  \item nonlinear,
  \item piecewise-defined,
  \item discontinuous.
\end{itemize}

\subsection{Tension Function}
\label{subsec:appendix-params-tension}

Structural tension is defined as:
\[
  T : S(K)\to\mathbb{R},
\]
with the semantics:
\[
  T<0 \Rightarrow s\in \Omega(K), \qquad
  T=0 \Rightarrow s\in \partial\Omega(K), \qquad
  T>0 \Rightarrow s\notin\Omega(K).
\]

Tension may depend on potentials and flows:
\[
  T = f(P, J, \Pi).
\]

\subsection{Composite Thresholds}
\label{subsec:appendix-params-composite}

Some transitions require multiple thresholds simultaneously:
\[
  T_1 \le \Theta_1,\quad
  T_2 \le \Theta_2,\quad
  \dots,\quad
  T_m \le \Theta_m.
\]

The admissible region then satisfies:
\[
  \Omega(K)=\{s:\max_k (T_k(s)-\Theta_k(s))\le0\}.
\]

\subsection{Parameter Evolution}
\label{subsec:appendix-params-evolution}

Parameters may evolve:
\[
  \frac{d\Pi}{dt} = W(A,P,\Theta,J,C,k,\Pi,t)
\]
where \(W\) is an operator describing:

\begin{itemize}
  \item adaptation,
  \item learning,
  \item evolution,
  \item institutional change,
  \item semantic drift.
\end{itemize}

\subsection{Landscape Interpretation}
\label{subsec:appendix-params-landscape}

Thresholds and tension define a landscape:
\[
  L(s) = T(s) - \Theta(s;\Pi).
\]
The sign determines position relative to \(\Omega\).

Phase transitions occur when the landscape touches zero.

\subsection{Examples by Level}
\label{subsec:appendix-params-levels}

\paragraph{K2 (Percolation).}
\[
  \Theta_{\mathrm{dim}} = p_c,\quad
  T = p - p_c.
\]

\paragraph{K3 (Autocatalytic Chemistry).}
\[
  \Theta_{\mathrm{react}}=0, \quad
  T = \Delta G.
\]

\paragraph{K4 (Protocells).}
\[
  T = \Pi_{\mathrm{osm}} - \Pi_{\mathrm{mem}},
\]
with \(\Pi_{\mathrm{mem}}\) determined by membrane parameters.

\paragraph{K5 (Neurons).}
\[
  \Theta_{\mathrm{spike}} = V_{\mathrm{thr}},
  \qquad T = V - V_{\mathrm{thr}}.
\]

\paragraph{K6 (Cognition).}
\[
  \Theta_{\mathrm{cog}} = \Theta_{\mathrm{PE}}, \quad
  T = D(m) - \Theta_{\mathrm{PE}}.
\]

\paragraph{K7 (Social Systems).}
\[
  T = T_{\mathrm{comm}} - \Theta_{\mathrm{coop}}.
\]

\paragraph{K8 (Civilizations).}
\[
  T = T_{\mathrm{inst}} - \Theta_{\mathrm{stab}}.
\]

\paragraph{K9–K12 (Metatheory).}
\[
  T = \text{logical or semantic inconsistency measure}.
\]

\subsection{Compact Summary}
\label{subsec:appendix-params-summary}

This appendix defines all parameter sets, tension functions, and thresholds
across levels. These determine:

\begin{itemize}
  \item the shape of \(\Omega(K)\),
  \item the location of \(\partial\Omega(K)\),
  \item all phase transitions,
  \item the emergence of new axes,
  \item and collapse conditions.
\end{itemize}

The formalism is universal across physics, chemistry, biology, cognition,
social systems, civilizations, and metatheoretical levels.
