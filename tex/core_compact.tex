% core_compact.tex
% Core v2.5 — Ultra-Compact Core Model

\section{The Compact Core Model}
\label{sec:core-compact}

This section presents the ultra-compact formulation of the Continuum Core
as fixed in Core~v2.5. It provides the minimal, level-independent structure
from which all continua \(K_0,\dots,K_{12}\) are obtained.

\subsection{Continuum Template}
\label{subsec:core-template}

A continuum \(K\) is defined by the tuple
\[
  K = \bigl(\Omega, A, P, \Theta, J, C, k\bigr)
\]
together with a state space \(S(K)\).
The components satisfy:

\begin{itemize}
  \item \textbf{Axes \(A\).}  
    A finite or countable family of axes \(A_j : S(K) \to \mathcal{A}_{A_j}\)
    giving distinguishable coordinates.

  \item \textbf{Potentials \(P\).}  
    Functions \(P_i : S(K) \to \mathcal{D}_{P_i}\)
    determining energetic, chemical, mechanical, cognitive,
    or abstract potentials.

  \item \textbf{Thresholds \(\Theta\).}  
    A set of critical values separating regimes of stability, transition, and
    collapse.

  \item \textbf{Flows \(J\).}  
    Mappings describing transport, transformation, influence,
    communication, or propagation.

  \item \textbf{Cycles \(C\).}  
    Closed trajectories \(C_n : [0,\tau_n] \to S(K)\)
    sustaining non-zero continuity.

  \item \textbf{Continuumness \(k\).}  
    A scalar function \(k : \Omega \to [0,1]\)
    measuring global connectedness and viability.
\end{itemize}

\subsection{Admissible Region}
\label{subsec:core-omega}

The admissible state set is
\begin{align*}
  \Omega(K) = \bigl\{\, s \in S(K) \ \big|\ &
    P_i(s) \in \mathcal{D}_{P_i},\
    A_j(s) \in \mathcal{A}_{A_j},\
    C(s) \neq \varnothing,\\
  & \vec{\Theta}(s) \ge 0,\
    T(s) \le 0 \bigr\},
\end{align*}
where \(T(s)\) is structural tension.
The boundary of admissible states is
\[
  \partial\Omega(K)
  = \bigl\{\, s \in S(K) \mid
    T(s)=0\ \text{or}\ 
    P_i(s) \in \partial \mathcal{D}_{P_i}\ \text{or}\ 
    A_j(s) \in \partial \mathcal{A}_{A_j}
  \bigr\}.
\]

\subsection{Core Dynamics}
\label{subsec:core-dynamics}

The universal evolution operator is
\[
  E : K(t) \mapsto K(t+\Delta t)
\]
with dynamics
\begin{align*}
  \frac{\mathrm{d}A}{\mathrm{d}t} &= F(A,P,\Theta,J,C,t),\\
  \frac{\mathrm{d}P}{\mathrm{d}t} &= G(A,P,\Theta,J,C,t),\\
  \frac{\mathrm{d}\Theta}{\mathrm{d}t} &= H(A,P,\Theta,J,C,t),\\
  \frac{\mathrm{d}J}{\mathrm{d}t} &= Q(A,P,\Theta,J,C,t),\\
  \frac{\mathrm{d}C}{\mathrm{d}t} &= S(A,P,\Theta,J,C,t),\\
  \frac{\mathrm{d}k}{\mathrm{d}t} &= U(A,P,\Theta,J,C,k,t).
\end{align*}

\subsection{Minimal Time Scale}
\label{subsec:core-time}

If at least one non-trivial cycle exists, the characteristic time scale is
\[
  \tau(K) = \min_n \tau_n.
\]
If \(C(K) = \varnothing\), then \(k=0\) and \(K\) is dead.

\subsection{Universal Axioms}
\label{subsec:core-axioms}

\paragraph{Axiom A1 (Difference).}
A continuum exists only if there are at least two distinguishable states.

\paragraph{Axiom A2 (Continuity).}
Continuumness is non-zero if and only if at least one closed cycle persists.

\paragraph{Axiom A3 (Threshold Constraint).}
A state is admissible only if all thresholds are satisfied,
i.e.\ \(T(s) \le 0\).

\paragraph{Axiom A4 (Conjugate Hierarchy).}
Each continuum \(K_x\) is constrained by its metaspace \(M_x\);
the next metaspace \(M_{x+1}\) is induced by \(K_x\).

\subsection{Global Theorems}
\label{subsec:core-theorems}

We summarise the universal theorems governing all continua.

\paragraph{Theorem T1 (Monotonicity of Dimension).}
If \(k(t)>0\), then \(\dim K(t)\) cannot decrease.

\paragraph{Theorem T2 (Death as Loss of Cycles).}
A continuum dies if and only if \(C=\varnothing\).

\paragraph{Theorem T3 (Compatibility with Metaspace).}
A continuum exists if and only if its admissible region is contained
in that of its metaspace:
\[
  \Omega(K_x) \subseteq \Omega(M_x).
\]

\subsection{Compact Summary}
\label{subsec:core-summary}

All continua \(K_0,\dots,K_{12}\) are instances of the universal template
\[
  K = (\Omega,A,P,\Theta,J,C,k),
\]
with level-specific instantiations of state space, axes, potentials,
thresholds, flows, cycles, and continuity.
This minimal core fully determines emergence, growth, collapse, and
inter-level transitions throughout the hierarchy.
