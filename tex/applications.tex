% applications.tex
% Core v2.5 — Applications and Cross-Level Predictions (Harmonised Extended Version)

\section{Applications and Cross-Level Predictions}
\label{sec:applications}

The Continuum Framework provides a unified structural language for analysing
physical, chemical, biological, cognitive, social, civilisational and
metatheoretical systems.  
Although each domain has its own content and mechanisms, their continua share
a common architecture governed by admissible sets, thresholds, cycles and
continuumness:
\[
  K = (\Omega, A, P, \Theta, J, C, k).
\]
This section presents a harmonised set of cross-level applications that extend
the compact predictions given in earlier sections but remain within the style
of the Preprint.  
Where appropriate, we provide explicit structural predictions and falsifiable
consequences.

\newpage
% ===============================================================
\subsection{Percolation and Critical Connectivity ($K_2$)}
% ===============================================================

Percolation is the canonical example of a continuum governed by a single
control parameter \(p\) and a sharp admissibility threshold:
\[
  T = p - p_c, \qquad
  p_c = \Theta_{\mathrm{dim}}.
\]
Continuumness is the probability of belonging to the infinite cluster:
\[
  k_2(p) = P_\infty(p).
\]

\paragraph{Prediction 1: Divergence of cycle periods.}
As \(p\to p_c\), the minimal cycle period diverges:
\[
  \tau(K_2) \to +\infty.
\]
Thus, any system whose macroscopic behaviour depends on long-range
connectivity---from magnetic systems to epidemic networks---exhibits critical
slowing down as it approaches the boundary \(\partial\Omega(K_2)\).

\paragraph{Prediction 2: Universal boundary curvature.}
Near the threshold:
\[
  k_2(p) \sim (p - p_c)^\beta,
\]
a testable prediction linking structural tension to known critical exponents.

\newpage
% ===============================================================
\subsection{Autocatalytic Networks and Chemical Closure ($K_3$)}
% ===============================================================

Autocatalytic reaction networks are continua where admissibility requires
the existence of a closed RAF (Reflexively Autocatalytic and F-Generated)
subnetwork.

\paragraph{Structural condition for closure.}
An RAF exists when:
\[
  \rho_{\mathrm{cat}} \cdot \rho_{\mathrm{closure}}
  \ge \Theta_{\mathrm{kas}},
\]
where \(\rho_{\mathrm{cat}}\) is catalytic coverage and
\(\rho_{\mathrm{closure}}\) is food-set accessibility.

\paragraph{Prediction 3: Sensitivity to single-reaction loss.}
If any essential reaction is removed,
\[
  \Omega(K_3) \to \varnothing,
\]
implying collapse of the autocatalytic structure.  
This is a falsifiable criterion for minimal chemical life.

\paragraph{Prediction 4: Catalytic diversity lowers tension.}
Increasing the diversity of catalysts decreases
\(T_3 = \Delta G_{\mathrm{act}} - \Delta G_{\mathrm{cat}}\)
and enlarges the admissible region \(\Omega(K_3)\).

\newpage
% ===============================================================
\subsection{Protocellular Stability and Boundary Formation ($K_4$)}
% ===============================================================

Transition \(K_3 \to K_4\) marks the emergence of a membrane and a new
dimension:
\[
  A_{\mathrm{in/out}} \in A(K_4).
\]

\paragraph{Structural condition for membrane viability.}
A protocell survives if:
\[
  | \Delta\pi | \le \Theta_{\mathrm{osm}},
  \qquad
  T_{\mathrm{mem}} \le \Theta_{\mathrm{mem}}.
\]

\paragraph{Prediction 5: Linear approach to rupture.}
As osmotic tension approaches the membrane threshold,
\[
  \tau_{\mathrm{rupture}} \propto
  (\Theta_{\mathrm{mem}} - T_{\mathrm{mem}})^{-1}.
\]
This predicts sudden collapse after prolonged marginal stress.

\paragraph{Prediction 6: Minimal metabolic cycle.}
Protocells require at least one stable metabolic cycle:
\[
  C_4 \neq \varnothing \quad \Rightarrow \quad k_4 > 0.
\]
This provides a model-independent viability condition for early life.

\newpage
% ===============================================================
\subsection{Neuronal Excitation and Spike Cycles ($K_5$)}
% ===============================================================

Neural continua introduce an excitability axis and a spike threshold:
\[
  T_5 = V - V_{\mathrm{thr}} = V - \Theta_{\mathrm{spike}}.
\]

\paragraph{Prediction 7: Universal spike onset.}
For all biophysically realistic neurons:
\[
  T_5 \ge 0 \quad \Rightarrow \quad
  \text{spike initiation}.
\]
Thus spiking is a boundary-crossing phenomenon: a transition across
\(\partial\Omega(K_5)\).

\paragraph{Prediction 8: Cycle failure predicts pathological states.}
If the spike cycle cannot close (e.g.\ depolarisation block):
\[
  C_{\mathrm{spike}} = \varnothing
  \quad\Rightarrow\quad
  k_5 \to 0.
\]
A formal explanation of seizures and runaway excitability follows from the
collapse of cycles.

\paragraph{Prediction 9: Conductance--frequency relation.}
Increasing sodium conductance reduces tension \(T_5\) and shortens
\(\tau_{\mathrm{spike}}\), predicting the universal direction of
frequency changes under channel modulation.

\newpage
% ===============================================================
\subsection{Cognitive Systems and Predictive Thresholds ($K_6$)}
% ===============================================================

Cognitive continua are defined by internal models, prediction errors and
attractor cycles.

\paragraph{Structural tension.}
\[
  T_6 = D(m) - \Theta_{\mathrm{PE}},
\]
where \(D(m)\) is the prediction discrepancy.

\paragraph{Prediction 10: Cognitive overload.}
When:
\[
  D(m) > \Theta_{\mathrm{PE}},
\]
the model becomes inadmissible and collapses, reducing \(k_6\).  
This formalises disorganisation, confusion and model failure.

\paragraph{Prediction 11: Minimal model-connectivity for coherence.}
Coherent cognition requires:
\[
  |C_{\max}^{(6)}| \ge \alpha |\mathcal{M}|,
  \quad \alpha\in[0.2,0.4].
\]
Below this threshold, thought fragments into disconnected subsystems.

\paragraph{Prediction 12: Alignment window for artificial agents.}
Agents remain aligned if:
\[
  D(m_{\mathrm{agent}}) \le \Theta_{\mathrm{PE}}.
\]
Misalignment is a structural boundary violation.

\newpage
% ===============================================================
\subsection{Social Systems, Norms and Institutional Cycles ($K_7$)}
% ===============================================================

Social continua require role structures, communication flows and
institutional cycles.

\paragraph{Prediction 13: Institutional collapse as cycle loss.}
If any stage of the institutional cycle fails:
\[
  C_{\mathrm{inst}} = \varnothing
  \quad\Rightarrow\quad
  k_7 = 0.
\]

\paragraph{Prediction 14: Communication asymmetry induces instability.}
If communication flows diminish:
\[
  J_{\mathrm{comm}} \downarrow \quad \Rightarrow \quad k_7 \downarrow.
\]
Polarisation is predicted as a monotonic function of communication failure.

\paragraph{Prediction 15: Hysteresis in recovery.}
Even after tension decreases, cycles may not immediately reform.
Thus the system exhibits hysteresis analogous to phase transitions with
memory.

\newpage
% ===============================================================
\subsection{Civilisational Thresholds and Long-Wave Dynamics ($K_8$)}
% ===============================================================

Civilisational continua combine infrastructure, institutions and
technological systems.

\paragraph{Structural tension.}
\[
  T_8 = T_{\mathrm{inst}} - \Theta_{\mathrm{stab}}.
\]

\paragraph{Prediction 16: Civilisational collapse.}
When infrastructural load exceeds capacity:
\[
  F_{\mathrm{load}} > \Theta_{\mathrm{cap}}
  \quad\Rightarrow\quad
  k_8\to 0.
\]

\paragraph{Prediction 17: Long-wave oscillations.}
Interactions of cycles \(C_8\) generate long temporal waves:
\[
  \tau_{C_8} \sim \tau_{\mathrm{innovation}}.
\]
This structurally unifies socio-economic long waves with neural and
metabolic rhythms.

\newpage
% ===============================================================
\subsection{Metatheoretical, Semantic and Integrative Continua ($K_9$--$K_{12}$)}
% ===============================================================

Higher continua describe theories, metamodels, semantic systems and
global integrative structures.

\paragraph{Prediction 18: Paradigm shift criterion ($K_9$).}
A paradigm changes when empirical anomaly tension exceeds a threshold:
\[
  T_9 \ge \Theta_{\mathrm{shift}}.
\]

\paragraph{Prediction 19: Functorial coherence threshold ($K_{10}$).}
Metamodels remain stable if functorial compatibility satisfies:
\[
  \mathrm{coh} \ge \Theta_{\mathrm{coh}}.
\]

\paragraph{Prediction 20: Semantic drift and collapse ($K_{11}$).}
If ambiguity exceeds the semantic threshold:
\[
  T_{11} > \Theta_{\mathrm{sem}}
  \quad\Rightarrow\quad k_{11}\to 0.
\]

\paragraph{Prediction 21: Global integration ($K_{12}$).}
The meta-integrative continuum exists only if:
\[
  C_{12}\neq\varnothing
  \quad\text{and}\quad
  \Omega(K_{12})\neq\varnothing.
\]

\newpage
% ===============================================================
\subsection{Compact Cross-Level Analogies}
% ===============================================================

Across all continua \(K_0\)–\(K_{12}\), the Framework predicts that:
\begin{itemize}
  \item phase transitions occur at structural thresholds,
  \item cycle existence determines stability,
  \item continuumness serves as a universal order parameter,
  \item collapse is always loss of admissibility or loss of cycles,
  \item dimension arises only under tension exceeding
        \(\Theta_{\mathrm{dim}}\),
  \item cross-level mappings preserve admissibility via metaspace.
\end{itemize}

This structural unity allows importing results between domains and yields
a powerful predictive apparatus for complex systems.

