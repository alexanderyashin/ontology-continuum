% applications.tex
% Core v2.5 — Applications of the Continuum Framework

\section{Applications}
\label{sec:applications}

This section presents applications of the Continuum Framework across multiple
domains. Each application uses the universal template
\[
  K = (\Omega, A, P, \Theta, J, C, k)
\]
to derive predictions, analyse systems, or generate new modelling tools.

\subsection{Application Class I: Emergence Analysis}
\label{subsec:applications-emergence}

The framework predicts the conditions under which new structures, functions,
or dimensions emerge.

\paragraph{Emergence Criterion.}
A new structure arises when:
\[
  T \ge \Theta_{\mathrm{dim}} \quad \text{and} \quad
  \Omega(K_x) \subseteq \Omega(M_{x+1}).
\]

\paragraph{Examples.}
\begin{itemize}
  \item \textbf{Physical emergence:} Percolation cluster formation at \(p=p_c\).
  \item \textbf{Chemical emergence:} Formation of RAF networks.
  \item \textbf{Biological emergence:} Birth of excitability in protocells.
  \item \textbf{Cognitive emergence:} New internal models after prediction-error tension.
  \item \textbf{Social emergence:} Formation of stable roles/institutions.
  \item \textbf{Civilizational emergence:} Appearance of new technological axes.
\end{itemize}

\subsection{Application Class II: Stability and Collapse}
\label{subsec:applications-collapse}

The Continuum Framework gives explicit collapse criteria.

\paragraph{Collapse Criterion.}
\[
  C=\varnothing \quad \lor \quad \Omega=\varnothing
  \quad \Rightarrow \quad k=0.
\]

\paragraph{Examples.}
\begin{itemize}
  \item \textbf{Chemical collapse:} Osmotic rupture of protocells.
  \item \textbf{Biological collapse:} Neuron unable to maintain cycles.
  \item \textbf{Cognitive collapse:} Loss of coherent model network.
  \item \textbf{Social collapse:} Breakdown of communication cycles.
  \item \textbf{Civilizational collapse:} Failure of core infrastructure cycles.
\end{itemize}

\subsection{Application Class III: Threshold Landscapes}
\label{subsec:applications-thresholds}

Thresholds determine phase transitions in all continua.

\paragraph{Universal Threshold Pattern.}
\[
  T(s) = \Theta(s) \quad \Rightarrow \quad s \in \partial\Omega.
\]

\paragraph{Applications.}
\begin{itemize}
  \item \textbf{Physics:} Critical phenomena; BKT threshold.
  \item \textbf{Chemistry:} Reaction feasibility boundaries.
  \item \textbf{Biology:} Spike thresholds under changing gradients.
  \item \textbf{Cognition:} Prediction-error thresholds for model change.
  \item \textbf{Sociology:} Cooperation threshold \(\Theta_{\mathrm{coop}}\).
  \item \textbf{Civilizations:} Structural stability thresholds.
\end{itemize}

\subsection{Application Class IV: Time and Cycles}
\label{subsec:applications-time}

Time is defined by cycles.

\paragraph{Universal Time Definition.}
\[
  \tau(K) = \min_n \tau_n.
\]

Applications include:

\begin{itemize}
  \item \textbf{Physics:} Correlation time near criticality.
  \item \textbf{Chemistry:} Turnover time of metabolic cycles.
  \item \textbf{Biology:} Spike timing and rhythmicity.
  \item \textbf{Cognition:} Cycle duration of attractors.
  \item \textbf{Social systems:} Institutional cycle periods.
  \item \textbf{Civilizations:} Long waves of stability.
\end{itemize}

\subsection{Application Class V: Modelling and Prediction}
\label{subsec:applications-modelling}

The Continuum Framework can be used to construct predictive models.

\paragraph{Modelling Pipeline.}
Given a target system \(S\):

\begin{enumerate}
  \item Define the state space \(S(S)\).
  \item Identify axes \(A_j(S)\).
  \item Identify potentials \(P_i(S)\).
  \item Identify thresholds \(\Theta_k(S)\).
  \item Identify flows \(J_m(S)\).
  \item Identify cycles \(C_n(S)\).
  \item Compute \(k(S)\).
  \end{enumerate}

\paragraph{Applications.}
\begin{itemize}
  \item Predict whether a protocell will maintain gradient homeostasis.
  \item Predict when a neuronal system will shift into synchronisation.
  \item Predict social collapse from weakening communication cycles.
  \item Predict civilizational instability from threshold violations.
\end{itemize}

\subsection{Application Class VI: Cross-Domain Mapping}
\label{subsec:applications-cross}

The Conjugate Hierarchy principle allows mappings:
\[
  K_x \leftrightarrow M_{x+1} \leftrightarrow K_{x+1}.
\]

\paragraph{Examples.}
\begin{itemize}
  \item Physical percolation maps to chemical connectivity.
  \item Osmotic thresholds map to excitability thresholds.
  \item Neuronal cycles map to cognitive cycles.
  \item Cognitive tension maps to social tension.
  \item Social stability maps to civilizational stability.
\end{itemize}

\subsection{Application Class VII: Multi-Level Diagnostics}
\label{subsec:applications-diagnostics}

Any system can be diagnosed across multiple levels simultaneously.

\paragraph{Diagnostic Map.}
\[
  \{A,P,\Theta,J,C,k\}_{x=0}^{12}.
\]

\paragraph{Use Cases.}
\begin{itemize}
  \item Identify which level constrains system performance.
  \item Detect hidden thresholds responsible for collapse.
  \item Predict which new axes must form for recovery.
\end{itemize}

\subsection{Compact Summary}
\label{subsec:applications-summary}

Applications of the Continuum Framework reduce to:

\begin{itemize}
  \item analysing emergence via dimensional thresholds,
  \item predicting collapse via loss of cycles,
  \item modelling phase transitions via structural tension,
  \item defining time via minimal cycles,
  \item constructing predictive models using the core template,
  \item mapping phenomena across disciplinary levels.
\end{itemize}

The result is a unified analytical tool capable of spanning physics,
chemistry, biology, cognition, social dynamics, civilizations,
and metatheoretical structures.
