% bibliography_structure.tex
% Core v2.5 — Bibliography Structure (placeholder for user-managed BibTeX)

\section{Bibliography Structure}
\label{sec:bibliography-structure}

This document uses an external BibTeX file to manage all references.
The purpose of this section is to define the structure and conventions
of the bibliography used throughout the Continuum Framework.

\subsection{BibTeX File Organisation}
\label{subsec:bibliography-organisation}

All references are stored in a dedicated file, for example:
\begin{verbatim}
references.bib
\end{verbatim}

Entries are grouped by discipline and by continuum level:
\begin{itemize}
  \item physics (K0–K2),
  \item chemistry (K2–K4),
  \item biology (K4–K6),
  \item cognition (K6),
  \item sociology (K7),
  \item civilization studies (K8),
  \item metatheory and metamodels (K9–K12),
  \item general systems theory, mathematics, logic.
\end{itemize}

\subsection{Citation Style}
\label{subsec:bibliography-style}

The recommended citation commands are:
\begin{itemize}
  \item \verb|\cite{key}| for parenthetical citations,
  \item \verb|\citep{key}| and \verb|\citet{key}| if using natbib,
  \item \verb|\textcite{key}| if using biblatex.
\end{itemize}

Entries should follow common BibTeX standards:
\begin{itemize}
  \item \verb|@article|,
  \item \verb|@book|,
  \item \verb|@inproceedings|,
  \item \verb|@misc| (for online or preprint materials).
\end{itemize}

\subsection{Naming Conventions}
\label{subsec:bibliography-naming}

Keys should follow the form:
\[
  \text{AuthorLastNameYearShortTitle}
\]
Examples:
\begin{verbatim}
Kauffman1993Origins
Szostak2018Protocells
Nassehi2005SozialeSysteme
Ladyman2013Complexity
Girvan2002Community
\end{verbatim}

\subsection{Cross-Level Referencing}
\label{subsec:bibliography-cross}

References may be associated with several levels. For example:
\begin{itemize}
  \item theoretical biology texts may be cited in both K4 and K5,
  \item computational neuroscience may sit between K5 and K6,
  \item sociocybernetics may link K7 and K8,
  \item logical or semantic works apply to K9–K12.
\end{itemize}

\subsection{BibTeX Example Entry}
\label{subsec:bibliography-example}

A typical BibTeX entry:
\begin{verbatim}
@article{Kauffman1993Origins,
  author    = {Stuart A. Kauffman},
  title     = {Autocatalytic Sets of Proteins},
  journal   = {Journal of Theoretical Biology},
  year      = {1993},
  volume    = {181},
  pages     = {47--94}
}
\end{verbatim}

\subsection{Summary}
\label{subsec:bibliography-summary}

This section defines the structure of the bibliography, expects references
to be supplied via an external BibTeX file, and outlines conventions for
consistent citation across all continuum levels in the unified hierarchy.
