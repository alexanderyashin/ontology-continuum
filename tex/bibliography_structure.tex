% bibliography_structure.tex
% Core v2.5 — Bibliography Structure

\section*{Bibliography}
\addcontentsline{toc}{section}{Bibliography}

% --------------------------------------------------------------------
% If you use BibTeX (recommended for the Continuum Project):
% --------------------------------------------------------------------

% Uncomment the following line once references.bib exists
% \bibliographystyle{plain}
% \bibliography{references}

% --------------------------------------------------------------------
% If you prefer manual entries (not recommended for long documents):
% --------------------------------------------------------------------

\begin{thebibliography}{99}

% --------------------------------------------------
% Physics (K0–K2)
% --------------------------------------------------
\bibitem{percolation}
D.~Stauffer and A.~Aharony.
\newblock {\em Introduction to Percolation Theory}.
\newblock Taylor and Francis, 1991.

\bibitem{bkt}
V.~L.~Berezinskii.
\newblock Destruction of Long-range Order in One-dimensional and
Two-dimensional Systems.
\newblock {\em Sov. Phys. JETP}, 32:493--500, 1971.

\bibitem{kosterlitz}
J.~M.~Kosterlitz and D.~J.~Thouless.
\newblock Ordering, metastability and phase transitions in two-dimensional
systems.
\newblock {\em J. Phys. C}, 6:1181--1203, 1973.

% --------------------------------------------------
% Chemistry (K3–K4)
% --------------------------------------------------
\bibitem{raf-eigen}
S.~A.~Kauffman.
\newblock Autocatalytic sets of proteins.
\newblock {\em J. Theor. Biol.}, 119:1--24, 1986.

\bibitem{szostak}
J.~W.~Szostak.
\newblock Protocells and the origin of cellular life.
\newblock {\em Annu. Rev. Biochem.}, 81:335--360, 2012.

% --------------------------------------------------
% Biology (K4–K5)
% --------------------------------------------------
\bibitem{hodgkin-huxley}
A.~L.~Hodgkin and A.~F.~Huxley.
\newblock A quantitative description of membrane current and its application
to conduction and excitation in nerve.
\newblock {\em J. Physiol.}, 117:500--544, 1952.

\bibitem{spike-initiation}
E.~M.~Izhikevich.
\newblock Simple model of spiking neurons.
\newblock {\em IEEE Trans. Neural Networks}, 14:1569--1572, 2003.

% --------------------------------------------------
% Cognition (K6)
% --------------------------------------------------
\bibitem{predictive-processing}
K.~Friston.
\newblock The free-energy principle: a unified brain theory?
\newblock {\em Nat. Rev. Neurosci.}, 11:127--138, 2010.

% --------------------------------------------------
% Social Systems (K7)
% --------------------------------------------------
\bibitem{luhmann}
N.~Luhmann.
\newblock {\em Social Systems}.
\newblock Stanford University Press, 1995.

% --------------------------------------------------
% Civilisation (K8)
% --------------------------------------------------
\bibitem{turchin}
P.~Turchin.
\newblock {\em Cliodynamics: The Journal of Quantitative History and Cultural
Evolution}, various articles.

% --------------------------------------------------
% Metatheory (K9–K11)
% --------------------------------------------------
\bibitem{kun}
T.~S.~Kuhn.
\newblock {\em The Structure of Scientific Revolutions}.
\newblock University of Chicago Press, 1962.

% --------------------------------------------------
% Artificial Systems (K12)
% --------------------------------------------------
\bibitem{distributed-ai}
B.~Grosz and S.~Kraus.
\newblock Collaborative multi-agent planning.
\newblock {\em AI Magazine}, 20:39--52, 1999.

\end{thebibliography}
