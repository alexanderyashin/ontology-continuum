% appendix_core.tex
% Core v2.5 — Appendix A: Core Mathematical Definitions

\section{Appendix A: Core Mathematical Definitions}
\label{sec:appendix-core}

This appendix expands the compact definition
\[
  K = (\Omega, A, P, \Theta, J, C, k)
\]
into precise mathematical structures consistent with the unified formulation
used throughout the Continuum Framework.

\subsection{State Space}
\label{subsec:appendix-core-state}

Each continuum \(K\) possesses a state space \(S(K)\), which may be continuous,
discrete, hybrid or symbolic.  
The \emph{admissible region} is a subset
\[
  \Omega(K)\subseteq S(K),
\]
defined by bounded tension and valid axis/potential domains.

\subsection{Axes of Difference}
\label{subsec:appendix-core-axes}

Axes represent independent structural distinctions:
\[
  A_j : S(K) \to \mathcal{A}_j.
\]
The full axis set is
\[
  A(K)=\{A_j\}_{j\in J_A}.
\]

\subsection{Potentials}
\label{subsec:appendix-core-potentials}

Potentials encode intensities, forces or resource levels:
\[
  P_i : S(K)\to \mathcal{D}_{P_i}.
\]
Domains \(\mathcal{D}_{P_i}\subseteq\mathbb{R}\) may include boundary points.

\subsection{Thresholds}
\label{subsec:appendix-core-thresholds}

Thresholds impose admissibility constraints:
\[
  \Theta_\ell : S(K)\to\mathbb{R}.
\]
The complete threshold vector is:
\[
  \vec{\Theta}=(\Theta_\ell)_{\ell\in J_\Theta}.
\]

\subsection{Flows}
\label{subsec:appendix-core-flows}

Flows describe transfers, interactions or causal propagation:
\[
  J_m : S(K)\times \mathbb{R}\to\mathcal{F}_m.
\]
All flows:
\[
  J(K)=\{J_m\}_{m\in J_J}.
\]

\subsection{Cycles}
\label{subsec:appendix-core-cycles}

Cycles are recurrent trajectories:
\[
  C_n : [0,\tau_n]\to S(K),\qquad C_n(0)=C_n(\tau_n).
\]
Cycles reflect structural self-maintenance.

\subsection{Continuumness}
\label{subsec:appendix-core-k}

Continuumness quantifies system viability:
\[
  k : S(K)\to[0,1],
\]
with
\[
  k=0 \iff \Omega(K)=\varnothing \ \text{or}\ C(K)=\varnothing.
\]

\subsection{Structural Tension}
\label{subsec:appendix-core-tension}

Tension encodes deviation from admissibility:
\[
  T : S(K)\to\mathbb{R}.
\]
Using tension components \(T_\ell\) and thresholds \(\Theta_\ell\),
\[
  T(s)=\max_\ell\bigl(T_\ell(s)-\Theta_\ell(s)\bigr).
\]
Thus:
\[
  T(s)\le0 \;\Rightarrow\; s\in\Omega(K),\qquad
  T(s)=0 \;\Rightarrow\; s\in\partial\Omega(K).
\]

\subsection{Boundary of Admissibility}
\label{subsec:appendix-core-boundary}

The boundary consists of states saturating constraints:
\[
  \partial\Omega(K)
  =
  \left\{
    s :
      T(s)=0
      \;\lor\;
      P_i(s)\in\partial\mathcal{D}_{P_i}
      \;\lor\;
      A_j(s)\in\partial\mathcal{A}_j
  \right\}.
\]

\subsection{Unified Evolution System}
\label{subsec:appendix-core-evolution}

Dynamics of a continuum are governed by the unified operator family
\((F,G,H,Q,S,U)\):
\begin{align*}
  \frac{dA}{dt} &= F(A,P,\Theta,J,C,t),\\
  \frac{dP}{dt} &= G(A,P,\Theta,J,C,t),\\
  \frac{d\Theta}{dt} &= H(A,P,\Theta,J,C,t),\\
  \frac{dT}{dt} &= Q(A,P,\Theta,J,C,T,t),\\
  \frac{dC}{dt} &= S(A,P,\Theta,J,C,t),\\
  \frac{dk}{dt} &= U(A,P,\Theta,J,C,T,k,t).
\end{align*}

The system evolves as
\[
  K(t)=K_0 + \int_0^t E(\tau)\,d\tau,
\]
where \(E\) collects the six componentwise operators above.

\subsection{Time Scale}
\label{subsec:appendix-core-time}

If cycles exist:
\[
  \tau(K)=\min_n \tau_n.
\]
If no cycles exist:
\[
  \tau(K)=+\infty,\qquad k=0.
\]

\subsection{Dimension}
\label{subsec:appendix-core-dimension}

Structural dimension is:
\[
  \dim K = |A(K)|.
\]

\subsection{Dimension Growth}
\label{subsec:appendix-core-dim-growth}

A new dimension appears during \(K_x\to K_{x+1}\) when:
\[
  T_x(s)\ge\Theta_x^{\mathrm{dim}}
  \quad\text{for some } s\in\Omega_x,
\]
and
\[
  \Omega_x,\Omega_{x+1}\subseteq\Omega(M_{x+1}).
\]
Then:
\[
  \dim K_{x+1} = \dim K_x + 1.
\]

\subsection{Collapse Conditions}
\label{subsec:appendix-core-collapse}

A continuum collapses if:
\[
  \Omega(K)=\varnothing
  \quad\text{or}\quad
  C(K)=\varnothing,
\]
or if
\[
  T(s)>\Theta(s)\quad\forall s\in S(K).
\]

\subsection{Compatibility with Metaspace}
\label{subsec:appendix-core-compatibility}

Existence requires:
\[
  \Omega(K_x)\subseteq \Omega(M_x),\qquad A(K_x)\subseteq A(M_x).
\]

\subsection{Compact Summary}
\label{subsec:appendix-core-summary}

This appendix formalises the core objects
\[
  S,\Omega,A,P,\Theta,J,C,k,T,\partial\Omega,\tau,\dim
\]
and the unified dynamics \((F,G,H,Q,S,U)\), forming the mathematical basis
for every level \(K_0\) through \(K_{12}\).
