% appendix_core.tex
% Core v2.5 — Appendix A: Core Mathematical Definitions

\section{Appendix A: Core Mathematical Definitions}
\label{sec:appendix-core}

This appendix provides the formal mathematical definitions underlying the
Continuum Framework. It expands the compact core specification
\[
  K = (\Omega, A, P, \Theta, J, C, k)
\]
into explicit structures suitable for mathematical analysis.

\subsection{State Space}
\label{subsec:appendix-core-state}

Each continuum \(K\) possesses a state space
\[
  S(K),
\]
which may be discrete, continuous, mixed, or categorical.
The admissible region is a subset \(\Omega(K)\subseteq S(K)\).

\subsection{Axes of Difference}
\label{subsec:appendix-core-axes}

Axes are measurable or distinguishable coordinates:
\[
  A_j : S(K) \to \mathcal{A}_{A_j}.
\]
The set of all axes is:
\[
  A(K) = \{ A_j \}_{j \in J_A}.
\]

\subsection{Potentials}
\label{subsec:appendix-core-potentials}

Potentials quantify forces or constraints:
\[
  P_i : S(K) \to \mathcal{D}_{P_i}.
\]

\subsection{Thresholds}
\label{subsec:appendix-core-thresholds}

Thresholds define admissibility boundaries:
\[
  \Theta_\ell : S(K) \to \mathbb{R}.
\]
All thresholds form a vector:
\[
  \vec{\Theta} = (\Theta_\ell)_{\ell\in J_\Theta}.
\]

\subsection{Flows}
\label{subsec:appendix-core-flows}

Flows describe transport or influence:
\[
  J_m : S(K)\times\mathbb{R} \to \mathcal{F}_m.
\]
Set of flows:
\[
  J(K) = \{J_m\}_{m\in J_J}.
\]

\subsection{Cycles}
\label{subsec:appendix-core-cycles}

Cycles are closed trajectories:
\[
  C_n : [0,\tau_n] \to S(K),
  \qquad C_n(0)=C_n(\tau_n).
\]

\subsection{Continuumness}
\label{subsec:appendix-core-k}

Continuumness is:
\[
  k : \Omega(K)\to[0,1].
\]

\subsection{Structural Tension}
\label{subsec:appendix-core-tension}

Structural tension is a scalar or vector field:
\[
  T : S(K)\to\mathbb{R},
\]
with sign indicating whether thresholds are satisfied:
\[
  T(s)\le0 \Rightarrow s\in\Omega(K), \qquad
  T(s)=0 \Rightarrow s\in\partial\Omega(K).
\]

\subsection{Boundary of Admissibility}
\label{subsec:appendix-core-boundary}

\[
  \partial\Omega(K) =
  \bigl\{ s \in S(K) \ \big|\ 
    T(s)=0 \lor 
    P_i(s)\in\partial\mathcal{D}_{P_i} \lor
    A_j(s)\in\partial\mathcal{A}_{A_j}
  \bigr\}.
\]

\subsection{Evolution Operator}
\label{subsec:appendix-core-evolution}

Dynamics are governed by:
\[
  E : K(t)\mapsto K(t+\Delta t),
\]
with component-wise updates:
\begin{align*}
  \frac{dA}{dt} &= F(A,P,\Theta,J,C,t), \\
  \frac{dP}{dt} &= G(A,P,\Theta,J,C,t), \\
  \frac{d\Theta}{dt} &= H(A,P,\Theta,J,C,t), \\
  \frac{dJ}{dt} &= Q(A,P,\Theta,J,C,t), \\
  \frac{dC}{dt} &= S(A,P,\Theta,J,C,t), \\
  \frac{dk}{dt} &= U(A,P,\Theta,J,C,k,t).
\end{align*}

\subsection{Time Scale}
\label{subsec:appendix-core-time}

If cycles exist:
\[
  \tau(K)=\min_n \tau_n.
\]
Otherwise:
\[
  \tau(K)=+\infty, \qquad k=0.
\]

\subsection{Dimension}
\label{subsec:appendix-core-dimension}

Dimension is defined via the number of independent axes:
\[
  \dim K = |A(K)|.
\]

\subsection{Dimension Growth}
\label{subsec:appendix-core-dim-growth}

A new dimension emerges when:
\[
  T\ge \Theta_{\mathrm{dim}}
  \quad\text{and}\quad
  \Omega(K_x)\subseteq\Omega(M_{x+1}).
\]

\subsection{Collapse Conditions}
\label{subsec:appendix-core-collapse}

Collapse occurs if:
\[
  \Omega=\varnothing
  \quad\text{or}\quad
  C=\varnothing.
\]

\subsection{Compatibility with Metaspace}
\label{subsec:appendix-core-compatibility}

For existence:
\[
  \Omega(K_x)\subseteq\Omega(M_x).
\]

\subsection{Compact Summary}
\label{subsec:appendix-core-summary}

This appendix formalises all core components:
\[
  S,\Omega,A,P,\Theta,J,C,k,T,\partial\Omega,E,\tau,\dim.
\]
These definitions provide the mathematical foundation of every level
\(K_0\) through \(K_{12}\).
