% appendix_figures.tex
% Core v2.5 — Appendix G: ASCII Figures

\section*{Appendix G: ASCII Figures}
\addcontentsline{toc}{section}{Appendix G: ASCII Figures}

This appendix provides schematic ASCII diagrams illustrating key structural
concepts of the Continuum Framework (Core v2.5). All figures are purely
textual and require no graphical packages.

% -------------------------------------------------------------
\subsection*{G.1 Universal Architecture of a Continuum}
\label{fig:universal_architecture}

\begin{verbatim}
                  +-------------------------------+
                  |           Continuum K         |
                  +-------------------------------+
                  |   Omega(K): admissible states |
                  |   A:      axes of difference  |
                  |   P:      potentials          |
                  |   Theta:  thresholds          |
                  |   J:      flows               |
                  |   C:      cycles              |
                  |   k(t):   continuumness       |
                  +-------------------------------+
\end{verbatim}

\bigskip

% -------------------------------------------------------------
\subsection*{G.2 Transition Operator Psi (Kx → Kx+1)}
\label{fig:transition_operator}

\begin{verbatim}
     +--------------+     Psi (Kx → Kx+1)     +---------------------+
     |     Kx       |  ---------------------> |    Kx_plus_1        |
     +--------------+                         +---------------------+
       Omega_x                                      Omega_x+1
       A_x                                          A_x+1
       P_x             new axis emerges             P_x+1
       Theta_x   ------------------------------->   Theta_x+1
       C_x                                          C_x+1
       k_x(t)                                       k_x+1(t)
\end{verbatim}

\bigskip

% -------------------------------------------------------------
\subsection*{G.3 K0 to K12 Ladder}
\label{fig:k_ladder}

\begin{verbatim}
 K12  Artificial meta-systems
 K11  Language and semantic systems
 K10  Meta-models and categories
 K9   Meta-theories and paradigms
 K8   Civilisational systems
 K7   Social systems and institutions
 K6   Cognitive systems
 K5   Neural systems
 K4   Cellular and protocellular systems
 K3   Autocatalytic chemistry
 K2   Percolation and critical fields
 K1   One-dimensional continuum
 K0   Meta-logical substrate
\end{verbatim}

\bigskip

% -------------------------------------------------------------
\subsection*{G.4 Admissible Region Ω(K) and Boundary ∂Ω(K)}
\label{fig:omega_boundary}

\begin{verbatim}
              +-------------------------------------+
              |            Omega(K)                 |
              |        (admissible states)          |
              |                                     |
              |   +-----------------------------+   |
              |   |      interior of Omega      |   |
              |   |   cycles C not empty        |   |
              |   +-----------------------------+   |
              |                                     |
              +-------------------------------------+
                       ^ boundary: dOmega(K)

Outside Omega(K): forbidden states
(thresholds exceeded or domains violated).
\end{verbatim}

\bigskip

% -------------------------------------------------------------
\subsection*{G.5 Generic Phase Transition of k(t)}
\label{fig:phase_transition}

\begin{verbatim}
                   T(t) < Theta    -> stable
                   T(t) = Theta    -> critical
                   T(t) > Theta    -> breakdown

      k(t)
       ^
     1 |---------\____
       |          \   \__ collapse
       +---------------------------> time or control parameter
\end{verbatim}

\bigskip

% -------------------------------------------------------------
\subsection*{G.6 Spike Cycle in Neural Continuum K5}
\label{fig:spike_cycle}

\begin{verbatim}
 Membrane potential V(t):

      +------------------ spike ------------------+
      |                                           |
 -70mV|___ depolarisation _/ \_ repolarisation ___|
      |                    \_/                    |
      +-------------------------------------------+
                threshold Theta_spike crossed

Phases:
- rest: stable potential
- depolarisation: V(t) crosses Theta_spike
- spike peak and repolarisation
- undershoot and return to rest

One full loop is the spike cycle with period tau_spike.
\end{verbatim}

\bigskip

% -------------------------------------------------------------
\subsection*{G.7 Institutional Cycle in Social Continuum K7}
\label{fig:institution_cycle}

\begin{verbatim}
     +-------------+
     | Norms       |
     +------^------+
            |
     +------v------+
     | Roles       |
     +------^------+
            |
     +------v------+
     | Expectations|
     +------^------+
            |
     +------v------+
     | Sanctions   |
     +-------------+

If any link breaks, the institutional cycle collapses and the
corresponding subsystem leaves Omega(K7).
\end{verbatim}

\bigskip

% -------------------------------------------------------------
\subsection*{G.8 Civilisational Tension Landscape in K8}
\label{fig:civ_tension}

\begin{verbatim}
    Tension T8
       ^
       |             /\    unstable region
       |            /  \
       |   stable  /    \   stable
       |__________/      \__________
               infrastructure scale or load ->
\end{verbatim}

\bigskip

% -------------------------------------------------------------
\subsection*{G.9 K3 → K4 Closure (Birth of the Membrane)}
\label{fig:k3k4_closure}

\begin{verbatim}
   +------------------------+      +----------------------------+
   |   RAF network (K3)     | ---> |  Bounded compartment (K4)  |
   +------------------------+      +----------------------------+
        reactions                        membrane
        autocatalysis                    transport rules
        simple cycles                    stable gradients
                                         membrane thresholds
\end{verbatim}

\bigskip

\section*{Summary of Appendix G}
\addcontentsline{toc}{subsection}{Summary of Appendix G}

The ASCII diagrams above illustrate the central constructions of the Continuum
Framework: universal continuum architecture, transition operators, the full
K0–K12 hierarchy, admissibility and boundary geometry, phase transitions of
continuity, neural and institutional cycles, civilisational tension, and the
birth of bounded compartments in the K3→K4 transition.
