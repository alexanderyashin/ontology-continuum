% appendix_figures.tex
% Core v2.5 — Appendix G: Figures, Diagrams, and Structural Maps
%
% Note: This appendix currently contains textual descriptions of the
% core figures. Diagram source code (TikZ or external graphics) can be
% added later without affecting the logical structure of the paper.

\section{Appendix G: Figures and Structural Maps}
\label{sec:appendix-figures}

This appendix documents the conceptual figures that accompany the
Continuum Framework.  
For the purposes of the GitHub build and automated compilation, we use
textual descriptions instead of in-document TikZ code.  
Diagram source files (for example \texttt{.tex} or \texttt{.pdf}) can be
maintained separately in a dedicated graphics directory.

\subsection{Figure 1: Universal Architecture of a Continuum}
\label{subsec:appendix-figures-architecture}

\textbf{Concept.}  
The first figure visualises the universal architecture of a continuum
\[
  K = \bigl( \Omega, A, P, \Theta, J, C, k \bigr).
\]

\textbf{Structure.}
\begin{itemize}
  \item Central node: the admissible set \(\Omega(K)\).
  \item Surrounding nodes: axes \(A\), potentials \(P\),
        thresholds \(\Theta\), flows \(J\), cycles \(C\),
        and continuumness \(k\).
  \item Arrows from \(A, P, \Theta, J, C, k\) into \(\Omega(K)\)
        represent that these components jointly determine which states
        belong to the admissible region.
  \item A separate node for the boundary \(\partial\Omega(K)\) with an
        arrow labelled ``critical'' indicates that the boundary is
        reached whenever structural tension hits the relevant thresholds.
\end{itemize}

\textbf{Intended use.}  
This figure provides an at-a-glance map of the universal template that
is instantiated at all levels \(K_0\) to \(K_{12}\).

\subsection{Figure 2: Transition Operator \texorpdfstring{\(\Psi_{x \rightarrow x+1}\)}{Psi}}
\label{subsec:appendix-figures-psi}

\textbf{Concept.}  
The second figure depicts the transition operator
\(\Psi_{x \rightarrow x+1}\) that maps one level \(K_x\)
to the next level \(K_{x+1}\) in the hierarchy.

\textbf{Structure.}
\begin{itemize}
  \item Two boxes labelled \(K_x\) and \(K_{x+1}\).
  \item A directed arrow from \(K_x\) to \(K_{x+1}\) labelled
        \(\Psi_{x \rightarrow x+1}\).
\end{itemize}

\textbf{Intended use.}  
The diagram emphasises that each level is not independent but is
generated via a structured transition that respects admissibility and
metaspace constraints.

\subsection{Figure 3: Hierarchy of Levels \texorpdfstring{\(K_0\)–\(K_{12}\)}{K0–K12}}
\label{subsec:appendix-figures-hierarchy}

\textbf{Concept.}  
The third figure shows the vertical hierarchy from the metalogical base
level \(K_0\) up to the meta-integrative level \(K_{12}\).

\textbf{Structure.}
\begin{itemize}
  \item A vertical stack of boxes labelled
        \(K_0, K_1, \dots, K_{12}\).
  \item Arrows from \(K_x\) to \(K_{x+1}\) for all
        \(x = 0,\dots,11\).
\end{itemize}

\textbf{Intended use.}  
This figure acts as the ``skeleton'' of the continuum hierarchy and is
referenced throughout the manuscript when discussing cross-level
relations and transitions.

\subsection{Figure 4: Boundary \texorpdfstring{\(\partial\Omega\)}{dOmega} as Threshold Surface}
\label{subsec:appendix-figures-boundary}

\textbf{Concept.}  
The fourth figure illustrates the admissible boundary
\(\partial\Omega(K)\) in an abstract state space.

\textbf{Structure.}
\begin{itemize}
  \item A smooth curve representing the boundary
        \(\partial\Omega(K)\).
  \item Sample points:
        \begin{itemize}
          \item A point clearly inside the boundary,
                labelled \(s \in \Omega(K)\).
          \item A point on the boundary,
                labelled \(s \in \partial\Omega(K)\).
        \end{itemize}
\end{itemize}

\textbf{Intended use.}  
This diagram supports the formal definition of admissible states and
their boundary and provides a geometric intuition for structural
tension and threshold conditions.

\subsection{Figure 5: Cycle Structure \texorpdfstring{\(C_n\)}{Cn}}
\label{subsec:appendix-figures-cycle}

\textbf{Concept.}  
The fifth figure visualises a generic cycle \(C_n\) in state space
with period \(\tau_n\).

\textbf{Structure.}
\begin{itemize}
  \item A closed curve (for example an elongated loop) indicating the
        trajectory of the system in state space.
  \item An arrow along the loop indicating direction of time.
  \item A label \(\tau_n\) indicating the period of the cycle.
\end{itemize}

\textbf{Intended use.}  
The figure underlines the role of cycles as fundamental carriers of
time and stability on all levels of the Continuum Framework.

\section*{Summary}
\label{sec:appendix-figures-summary}

This appendix defines the conceptual content of five core figures:

\begin{itemize}
  \item the universal architecture of a continuum,
  \item the transition operator \(\Psi_{x \rightarrow x+1}\),
  \item the full hierarchy \(K_0\)–\(K_{12}\),
  \item the admissibility boundary \(\partial\Omega\),
  \item and a generic cycle \(C_n\) with period \(\tau_n\).
\end{itemize}

For automated builds and textual review, these descriptions are
sufficient. High-resolution diagrams (for example in TikZ or vector
graphics) can be added to the publication package at a later stage
without changing the logical structure of the document.
