% appendix_figures.tex
% Core v2.5 — Appendix G: ASCII Figures

\section*{Appendix G: ASCII Figures}
\addcontentsline{toc}{section}{Appendix G: ASCII Figures}

This appendix contains schematic ASCII diagrams illustrating the structure
of the Continuum Framework (Core v2.5). All figures are text only and
require no additional graphics packages.

% -------------------------------------------------------------
\subsection*{G.1 Universal Architecture of a Continuum}
\label{fig:universal_architecture}

\begin{verbatim}
                  +-------------------------------+
                  |           Continuum K         |
                  +-------------------------------+
                  |   Omega(K): admissible states |
                  |   A:      axes of difference  |
                  |   P:      potentials          |
                  |   Theta:  thresholds          |
                  |   J:      flows               |
                  |   C:      cycles              |
                  |   k(t):   continuity measure  |
                  +-------------------------------+
\end{verbatim}

\bigskip

% -------------------------------------------------------------
\subsection*{G.2 Transition Operator Psi (Kx to Kx+1)}
\label{fig:transition_operator}

\begin{verbatim}
     +--------------+     Psi (Kx to Kx+1)   +---------------------+
     |   Kx         |  ------------------->  |   Kx_plus_1          |
     +--------------+                        +---------------------+
       Omega_x                                   Omega_x_plus_1
       A_x                                       A_x_plus_1
       P_x            Birth of new axis          P_x_plus_1
       Theta_x  ----------------------------->   Theta_x_plus_1
       C_x                                       C_x_plus_1
       k_x(t)                                    k_x_plus_1(t)
\end{verbatim}

\bigskip

% -------------------------------------------------------------
\subsection*{G.3 K0 to K12 Ladder}
\label{fig:k_ladder}

\begin{verbatim}
 K12  Artificial meta-systems
 K11  Language and semantic systems
 K10  Meta-models and categories
 K9   Meta-theories and paradigms
 K8   Civilisational systems
 K7   Social systems and institutions
 K6   Cognitive systems
 K5   Neural systems
 K4   Cellular and protocellular systems
 K3   Autocatalytic chemistry
 K2   Percolation and critical fields
 K1   One-dimensional continuum
 K0   Meta-logical substrate
\end{verbatim}

\bigskip

% -------------------------------------------------------------
\subsection*{G.4 Admissible Region Omega(K) and Boundary}
\label{fig:omega_boundary}

\begin{verbatim}
              +-------------------------------------+
              |            Omega(K)                 |
              |       (admissible states)           |
              |                                     |
              |   +-----------------------------+   |
              |   |     interior of Omega       |   |
              |   |   cycles C not empty        |   |
              |   +-----------------------------+   |
              |                                     |
              +-------------------------------------+
                       ^ boundary: dOmega(K)

Outside Omega(K): forbidden states (thresholds exceeded or
axes / potentials outside their allowed domains).
\end{verbatim}

\bigskip

% -------------------------------------------------------------
\subsection*{G.5 Generic Phase Transition of Continuity k(t)}
\label{fig:phase_transition}

\begin{verbatim}
                   T(t) < Theta    -> stable
                   T(t) = Theta    -> critical
                   T(t) > Theta    -> breakdown

      k(t)
       ^
     1 |---------\____
       |          \   \__ collapse
       +--------------------------->
                     control parameter or time
\end{verbatim}

\bigskip

% -------------------------------------------------------------
\subsection*{G.6 Spike Cycle in Neural Continuum K5}
\label{fig:spike_cycle}

\begin{verbatim}
 Membrane potential V(t):

      +------------------ spike ------------------+
      |                                           |
 -70mV|___ depolarisation _/ \_ repolarisation ___|
      |                    \_/                    |
      +-------------------------------------------+
             threshold Theta_spike crossed

Phases:
- rest: stable membrane potential
- depolarisation: V(t) crosses Theta_spike
- spike peak and repolarisation
- undershoot and return to rest

One full loop is the spike cycle with period tau_spike.
\end{verbatim}

\bigskip

% -------------------------------------------------------------
\subsection*{G.7 Institutional Cycle in Social Continuum K7}
\label{fig:institution_cycle}

\begin{verbatim}
     +-------------+
     | Norms       |
     +------^------+
            |
     +------v------+
     | Roles       |
     +------^------+
            |
     +------v------+
     | Expectations|
     +------^------+
            |
     +------v------+
     | Sanctions   |
     +-------------+

If any arrow is broken, the institutional cycle fails and the
corresponding part of the social continuum leaves its admissible set.
\end{verbatim}

\bigskip

% -------------------------------------------------------------
\subsection*{G.8 Civilisational Tension Landscape in K8}
\label{fig:civ_tension}

\begin{verbatim}
    Tension T8
       ^
       |             /\   unstable region
       |            /  \
       |   stable  /    \   stable
       |__________/      \__________
               infrastructure scale or load ->
\end{verbatim}

\bigskip

% -------------------------------------------------------------
\subsection*{G.9 K3 to K4 Closure (Birth of Membrane)}
\label{fig:k3k4_closure}

\begin{verbatim}
   +------------------------+      +----------------------------+
   |   RAF network (K3)     | ---> |  Bounded compartment (K4)  |
   +------------------------+      +----------------------------+
          reactions                      membrane
          autocatalysis                  transport rules
          simple cycles                  stable gradients
                                         membrane thresholds
\end{verbatim}

\bigskip

\section*{Summary of Appendix G}
The ASCII diagrams above provide visual intuition for the main
constructions of the Continuum Framework: the universal architecture
of a continuum, dimensional transitions, the K0 to K12 hierarchy,
admissible regions and boundaries, generic phase transitions,
neural and institutional cycles, civilisational tension, and the
closure step from K3 to K4.
