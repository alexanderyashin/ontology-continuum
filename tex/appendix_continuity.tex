% appendix_continuity.tex
% Core v2.5 — Appendix B: Continuity and Transition Operators

\section{Appendix B: Continuity and Transition Operators}
\label{sec:appendix-continuity}

This appendix provides the detailed mathematical formalisation of continuity
conditions, dimensional transitions, and collapse criteria used to define
inter-level dynamics in the Continuum Framework.

\subsection{Transition Operators}
\label{subsec:appendix-continuity-psi}

Each transition \(K_x \to K_{x+1}\) is generated by a transition operator
\[
  \Psi_{x\rightarrow x+1} : K_x \longrightarrow K_{x+1}
\]
acting on all components:
\[
  \Psi_{x\rightarrow x+1}(\Omega, A, P, \Theta, J, C, k)
  =
  (\Omega', A', P', \Theta', J', C', k').
\]

\subsection{Continuity Conditions}
\label{subsec:appendix-continuity-cond}

A transition is \emph{continuous} if the following hold:

\paragraph{(C1) Admissibility Preservation.}
\[
  \Omega(K_x) \subseteq \Omega(M_{x+1}).
\]

\paragraph{(C2) Axis Extension.}
\[
  A(K_x) \subseteq A(K_{x+1}), \qquad
  A(K_{x+1}) = A(K_x)\cup\{A_{\mathrm{new}}\}.
\]

\paragraph{(C3) Threshold Inheritance.}
\[
  \Theta(K_x)\subseteq\Theta(K_{x+1})
\]
and new thresholds correspond only to new axes.

\paragraph{(C4) Cycle Embedding.}
Every cycle in \(K_x\) persists as a sub-cycle in \(K_{x+1}\).

\paragraph{(C5) Continuity Preservation.}
\[
  k_x>0 \implies k_{x+1}>0.
\]

\subsection{Dimension Emergence}
\label{subsec:appendix-continuity-dim}

A new axis appears only when the following criteria are satisfied:

\paragraph{(D1) Structural Incompatibility.}
Existing axes cannot represent a new class of differences.

\paragraph{(D2) Dimensional Tension.}
\[
  T_x \ge \Theta_x^{\mathrm{dim}}.
\]

\paragraph{(D3) Metaspace Admissibility.}
\[
  \Omega(K_x) \subseteq \Omega(M_{x+1})
  \quad\text{and}\quad
  \Omega(K_{x+1}) \subseteq \Omega(M_{x+1}).
\]

If so:
\[
  \dim K_{x+1} = \dim K_x + 1.
\]

\subsection{Collapse Conditions}
\label{subsec:appendix-continuity-collapse}

Collapse occurs if any of the following hold:

\paragraph{(CL1) Loss of Admissibility.}
\[
  \Omega = \varnothing.
\]

\paragraph{(CL2) Loss of Cycles.}
\[
  C = \varnothing.
\]

\paragraph{(CL3) Threshold Violation.}
\[
  T(s) > \Theta(s) \ \forall s\in S(K).
\]

\paragraph{(CL4) Metaspace Violation.}
\[
  \Omega(K) \not\subseteq \Omega(M).
\]

In collapse:
\[
  k = 0.
\]

\subsection{Time Continuity}
\label{subsec:appendix-continuity-time}

Let the cycle periods of \(K_x\) be \(\{\tau_n^x\}\).
Then time is continuous across levels if:
\[
  \tau(K_{x+1}) \le \tau(K_x) \quad\text{or}\quad
  \tau(K_{x+1}) = \tau(K_x).
\]
If cycles expand without bound:
\[
  \tau(K_x)\to +\infty,
\]
the continuum is dead.

\subsection{Path Continuity in State Space}
\label{subsec:appendix-continuity-path}

A trajectory \(s(t)\) is continuous across levels if:
\[
  s(t)\in\Omega(K_x)
  \quad\text{and}\quad
  \Psi_{x\rightarrow x+1}(s(t))\in\Omega(K_{x+1})
\]
for all relevant \(t\).

\subsection{Metaspace Chain Conditions}
\label{subsec:appendix-continuity-meta}

The metaspace chain obeys:

\[
  \Omega(K_x)\subseteq\Omega(M_x), \qquad
  A(K_x)\subseteq A(M_x).
\]

Furthermore:
\[
  \mathrm{DoF}(K_x)\le \mathrm{DoF}(M_{x+1}).
\]

\subsection{Compact Summary}
\label{subsec:appendix-continuity-summary}

Transition operators formalise the hierarchy:

\[
  K_0 \xrightarrow{\Psi_{0\rightarrow1}} K_1
     \xrightarrow{\Psi_{1\rightarrow2}} K_2
     \dots
     \xrightarrow{\Psi_{11\rightarrow12}} K_{12}.
\]

Continuity requires:
\[
  \Omega_x\subseteq\Omega_{M_{x+1}},\ 
  A_x\subseteq A_{x+1},\ 
  C_x\subseteq C_{x+1},\ 
  k_x>0\Rightarrow k_{x+1}>0.
\]

Dimension grows monotonically, and collapse occurs only through loss of cycles
or violation of all thresholds.
